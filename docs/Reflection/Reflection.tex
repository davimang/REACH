
\documentclass{article}

\usepackage{tabularx}
\usepackage{booktabs}

\title{Reflection Report on \progname}

\author{\authname}

\date{}

%% Comments

\usepackage{color}

\newif\ifcomments\commentstrue %displays comments
%\newif\ifcomments\commentsfalse %so that comments do not display

\ifcomments
\newcommand{\authornote}[3]{\textcolor{#1}{[#3 ---#2]}}
\newcommand{\todo}[1]{\textcolor{red}{[TODO: #1]}}
\else
\newcommand{\authornote}[3]{}
\newcommand{\todo}[1]{}
\fi

\newcommand{\wss}[1]{\authornote{blue}{SS}{#1}} 
\newcommand{\plt}[1]{\authornote{magenta}{TPLT}{#1}} %For explanation of the template
\newcommand{\an}[1]{\authornote{cyan}{Author}{#1}}

%% Common Parts

\newcommand{\progname}{Software Engineering} % PUT YOUR PROGRAM NAME HERE
\newcommand{\authname}{Team 14, Reach
\\ Aamina Hussain
\\ David Moroniti
\\ Anika Peer
\\ Deep Raj
\\ Alan Scott} % AUTHOR NAMES                  

\usepackage{hyperref}
    \hypersetup{colorlinks=true, linkcolor=blue, citecolor=blue, filecolor=blue,
                urlcolor=blue, unicode=false}
    \urlstyle{same}
                                


\begin{document}

\maketitle

\plt{Reflection is an important component of getting the full benefits from a
    learning experience.  Besides the intrinsic benefits of reflection, this
    document will be used to help the TAs grade how well your team responded to
    feedback.  In addition, several CEAB (Canadian Engineering Accreditation Board)
    Learning Outcomes (LOs) will be assessed based on your reflections.}

\section{Changes in Response to Feedback}

\plt{Summarize the changes made over the course of the project in response to
    feedback from TAs, the instructor, teammates, other teams, the project
    supervisor (if present), and from user testers.}

\plt{For those teams with an external supervisor, please highlight how the feedback
    from the supervisor shaped your project.  In particular, you should highlight the
    supervisor's response to your Rev 0 demonstration to them.}

\plt{Version control can make the summary relatively easy, if you used issues
    and meaningful commits.  If you feedback is in an issue, and you responded in
    the issue tracker, you can point to the issue as part of explaining your
    changes.  If addressing the issue required changes to code or documentation, you
    can point to the specific commit that made the changes.}

\subsection{SRS and Hazard Analysis}

There were several changes made to the SRS and hazard analysis as a result of feedback from Dr. Smith, our TAs, peers, and supervisors.\\

SRS:

\begin{itemize}
    \item Peer feedback
          \begin{itemize}
              \item Update the individual product use cases to be fully black-box use cases. We did this by only focusing on what the use case is, and not describing how it could be done. This ensures the design is not constrained in the SRS (see \href{https://github.com/davimang/REACH/issues/12}{Issue 12}, \href{https://github.com/davimang/REACH/commit/7b80cb5d97721e50163f8874ba8e1d98c0bc45f9}{this commit} and \href{https://github.com/davimang/REACH/commit/916252afc158618ca8039c603c2f46ad836c0947}{this commit} for more details).
              \item Improve NFR-1 to make it less ambiguous (see \href{https://github.com/davimang/REACH/issues/14}{Issue 14} and \href{https://github.com/davimang/REACH/commit/6e514cf91ae0eae4c481530bf3403c99eb11770b}{this commit} for more details).
              \item Improve NFR-12 to make it less ambiguous (see \href{https://github.com/davimang/REACH/issues/15}{Issue 15} and \href{https://github.com/davimang/REACH/commit/6e514cf91ae0eae4c481530bf3403c99eb11770b}{this commit} for more details).
              \item Added user characteristics to make the target users more clear (see \href{https://github.com/davimang/REACH/issues/13}{Issue 13} and \href{https://github.com/davimang/REACH/commit/80c3ac1591a79c1b5f9850462cc52d9fbc6672e5}{this commit} for more details).
          \end{itemize}
    \item Dr. Smith and supervisors - After the revision 0 demo, both parties had very useful feedback, which we implemented in many places throughout the app and documentation.
          \begin{itemize}
              \item Add a map to be able to see exactly where the trial/study is being held (see \href{https://github.com/davimang/REACH/issues/138}{Issue 138} and \href{https://github.com/davimang/REACH/commit/8b6dd42a8c4585f0ac4a0efce1ea32024c65f6c0}{this commit} for more details). This was recommended by our supervisors.
              \item Add a disclaimer/title to the profile creation page to make the purpose of the page more clear (see \href{https://github.com/davimang/REACH/issues/155}{Issue 155} for more details). This was recommended by both Dr. Smith and our supervisors.
              \item Add the ability to filter the studies/trials by distance (see \href{https://github.com/davimang/REACH/issues/178}{Issue 178} and \href{https://github.com/davimang/REACH/commit/8b6dd42a8c4585f0ac4a0efce1ea32024c65f6c0}{this commit} for more details). This was recommended by both Dr. Smith and our supervisors.
          \end{itemize}
    \item Other
          \begin{itemize}
              \item Add FR-4, FR-5, FR-11 to the waiting room as they were not implemented.
          \end{itemize}
\end{itemize}

Some other useful issues to give more insight into the changes we have made to the SRS documentation are provided below:

\begin{itemize}
    \item \href{https://github.com/davimang/REACH/issues/251}{Issue 251}\\
\end{itemize}

Hazard Analysis:

\begin{itemize}
    \item Peer feedback/TA feedback
          \begin{itemize}
              \item Added some critical assumptions. This addresses feedback provided to us by our peers, and also feedback from the TA via the graded rubric (see \href{https://github.com/davimang/REACH/issues/35}{Issue 35} and \href{https://github.com/davimang/REACH/commit/643f128b7c4bd9c3072f56c813d5551e18c0af4e}{this commit} for more details).
              \item Added a definition of failure. This also addresses feedback provided to us by our peers and the TA (see \href{https://github.com/davimang/REACH/issues/36}{Issue 36} and \href{https://github.com/davimang/REACH/commit/3909aa4b28693cec5cc4ef3a9d7b84c41a6ffb24}{this commit} for more details).
              \item Add failure mode for email generation. This also addressed feedback provided to us by our peers and the TA (see \href{https://github.com/davimang/REACH/issues/36}{Issue 37} and \href{https://github.com/davimang/REACH/commit/e39f11a5c2fe7af0b6e261a353f818e2780ac43d}{this commit} for more details).
              \item Make SR-6 less ambiguous by improving the wording of the requirement. This addresses feedback provided to use by our peers (see \href{https://github.com/davimang/REACH/issues/39}{Issue 39} and \href{https://github.com/davimang/REACH/commit/bb0701657d9f217275ee0027e30f728f8924e277}{this commit} for more details).
              \item Removed the notification subsystem as it is currently out of scope. This addresses feedback provided to us by our peers (see \href{https://github.com/davimang/REACH/issues/38}{Issue 38} and \href{https://github.com/davimang/REACH/commit/605e299034c55c6f3bf19bac1514fdb3b63c489e}{this commit} for more details).
              \item Added traceability to FRs in the FMEA table (see \href{https://github.com/davimang/REACH/issues/40}{Issue 40} and \href{https://github.com/davimang/REACH/commit/7969438165b2ca71af4ee2ec926f6ef81b3a9c1b}{this commit} for more details).
          \end{itemize}
\end{itemize}

Some other useful issues to give more insight into the changes we have made to the Hazard Analysis documentation are provided below:

\begin{itemize}
    \item \href{https://github.com/davimang/REACH/issues/250}{Issue 250}
\end{itemize}


\subsection{Design and Design Documentation}
There was very little feedback given for this particular section of the project, 
we generally performed well in this area. However, the changes made will be denoted below: \newline

\noindent MG: 
\begin{itemize}
	\item Peer/TA Feedback.
	\begin{itemize}
        \item Added all Functional and Non-Functional requirements to the traceability matrix in response to TA feedback.
        \item Corrected some minor sentence structure issues in the document in response to TA feedback.
    \end{itemize}
\end{itemize}


MIS:
\begin{itemize}
    \item There was no feedback given for this deliverable, grammar was checked and minor flow changes were made. \newline
\end{itemize}


System Design:
\begin{itemize}
    \item There was no feedback given for this deliverable, grammar was checked and minor changes were made. \newline
\end{itemize}

VnV Plan:
\begin{itemize}
	\item Peer/TA Feedback.
	\begin{itemize}
		\item Added missing abbreviations to the symbols section in response to peer review (see \href{https://github.com/davimang/REACH/issues/45}{Issue 45} and \href{https://github.com/davimang/REACH/commit/ed1e1b779a09e4f9307eda85c31f5c630c82901a}{this commit} for more details).
		\item Added addition characteristics for the typical user tester in response to peer review (see \href{https://github.com/davimang/REACH/issues/46}{Issue 46} and \href{https://github.com/davimang/REACH/commit/fddfee5364754dc5f378b798aa95c6b2c8936416}{this commit} for more details).
		\item Added reference to the grading rubric in response to peer review (see \href{https://github.com/davimang/REACH/issues/47}{Issue 47} and \href{https://github.com/davimang/REACH/commit/88505588e5dba20dca6da5269ea3a9d1a4790b0b}{this commit} for more details).
		\item Added clarification on what a greater than 7 score would mean in terms of testing in response to peer review (see \href{https://github.com/davimang/REACH/issues/48}{Issue 48} and \href{https://github.com/davimang/REACH/commit/b0e38a2fd6e89f05ef0aa6bf5e46b53227dce24f}{this commit} for more details).
		\item Added an open ended additional concerns question to the usability survey in response to peer feedback (see \href{https://github.com/davimang/REACH/issues/49}{Issue 49} and \href{https://github.com/davimang/REACH/commit/10b7cbf6ab0c7e78344dd1dff75ae76adf6cd4c2}{this commit} for more details).
		\item Fixed some formatting issues including removing some whitespace from the document in response to TA feedback through the marking rubric (see \href{https://github.com/davimang/REACH/issues/255}{Issue 255} and \href{https://github.com/davimang/REACH/commit/ad4b67e9f87e89c4d2c1a409db14da7401b8f39c}{this commit} for more details).
		\item Improved clarity of NFR and FR tests in response to TA feedback through the marking rubric (see \href{https://github.com/davimang/REACH/issues/256}{Issue 256} and \href{https://github.com/davimang/REACH/commit/ce52fb62cb65dd884ee3c7afac4c7a684d84b771}{this commit} for more details).
		\item Adding missing reference to hazard analysis in the relevant documents section in response to TA feedback through the marking rubric (see \href{https://github.com/davimang/REACH/issues/257}{Issue 257} and \href{https://github.com/davimang/REACH/commit/396e5df0a6208758a45054c8e35b35d37d1cc25f}{this commit} for more details).
		\item Added more specific plan for nondynamic testing in response to TA feedback through marking rubric (see \href{https://github.com/davimang/REACH/issues/258}{Issue 258} and \href{https://github.com/davimang/REACH/commit/3a46bc35ff31855f752df09efb5f4dd5fa4211c7}{this commit} for more details).
	\end{itemize}
\end{itemize}

VnV Report:
\begin{itemize}
	\item TA Feedback (No peer review was received for this deliverable).
	\begin{itemize}
		\item Expanded on the changes made due to testing, in response to received feedback on the marking rubric (see \href{https://github.com/davimang/REACH/issues/254}{Issue 254} and \href{https://github.com/davimang/REACH/commit/06620ec4af39e47e90e0647ae05cda2a58971ab7}{this commit} for more details).3
		\item Added a screenshot of the failed linting in the maintainability section to provide convincing evidence of the test failing, in response to received feedback on the marking rubric (see \href{https://github.com/davimang/REACH/issues/253}{Issue 253} and \href{https://github.com/davimang/REACH/commit/06620ec4af39e47e90e0647ae05cda2a58971ab7}{this commit} for more details).
	\end{itemize}
\end{itemize}


\section{Design Iteration (LO11)}

\plt{Explain how you arrived at your final design and implementation.  How did
    the design evolve from the first version to the final version?}

We approached the design, development, and implementation of this project using an iterative approach. 
We started off by coming up with the project requirements, and then splitting them into two groups: 
requirements that we must have and requirements that we would like to have time-permitting. Taking a 
look at the must-have requirements, we realized that our entire project relied on whether or not we 
could actually use the clinicaltrials.gov API. Since this was a huge priority, we decided for the proof 
of concept demo we would prove we could successfully utilize the external API.\\

Once we were able to verify the API worked, we officially started the development phase of the project. 
We worked on implementing the must-have requirements to the best of our abilities. We did this by creating 
tasks (using issues in GitHub) that we needed to get done in order to fully implement each requirement. 
For each individual task, we made the necessary code changes, did some minor testing locally, made further 
based on the results of the tests, merged the changes, and closed the issue for the task. In this case, we 
were iteratively developing on a smaller scale (on a task-by-task basis).\\

Once we completed implementing all these smaller tasks in order to fully implement the must-have 
requirements, we deployed the app and conducted some more thorough and integrated testing. This testing 
also included conducting a usability test with multiple users; one of the users was our supervisor. 
Throughout testing, we made note of things that did not work, needed improvement, or was redundant. 
Taking these notes, along with user feedback we received from the usability survey and our supervisors, 
we came up with some more tasks that we needed to complete in order to reflect our notes and feedback. 
This was a large-scale development iteration (of the overall integrated application).\\

We then repeated the process once more by making necessary changes for each task, deploying the app, 
and once again fully testing the overall application. Gathering our last round of feedback, we made 
the necessary changes and deployed the app once more before our final demo. This was where we stopped 
for now, but in order to keep improving our app and dealing with any bugs that we may discover or 
suggestions from our supervisors, we would continuously repeat this iterative process.

\section{Design Decisions (LO12)}

\plt{Reflect and justify your design decisions.  How did limitations,
    assumptions, and constraints influence your decisions?}

\section{Economic Considerations (LO23)}

\plt{Is there a market for your product? What would be involved in marketing your
    product? What is your estimate of the cost to produce a version that you could
    sell?  What would you charge for your product?  How many units would you have to
    sell to make money? If your product isn't something that would be sold, like an
    open source project, how would you go about attracting users?  How many potential
    users currently exist?}

\section{Reflection on Project Management (LO24)}

\plt{This question focuses on processes and tools used for project management.}

\subsection{How Does Your Project Management Compare to Your Development Plan}

\plt{Did you follow your Development plan, with respect to the team meeting plan,
    team communication plan, team member roles and workflow plan.  Did you use the
    technology you planned on using?}

\subsection{What Went Well?}

\plt{What went well for your project management in terms of processes and
    technology?}

\subsection{What Went Wrong?}

\plt{What went wrong in terms of processes and technology?}

\subsection{What Would you Do Differently Next Time?}

\plt{What will you do differently for your next project?}

\section{Reflection on Capstone}

\plt{This question focuses on what you learned during the course of the capstone project.}

\subsection{Which Courses Were Relevant}

The following course provided beneficial experience towards the undertaking of our capstone project:
\begin{itemize}
	\item SFWRENG 3XA3 Software Project Management - Provided experience in coding development, but more importantly in managing the planning and documentation of a project from start to finish.
	\item SFWRENG 3S03 Software Testing - Provided knowledge of software testing strategies, such as static testing, unit testing, and code reviews.
	\item SFWRENG 3A04 Software Design III - Provided project development experience, specifically in module design and documentation.
	\item SFWRENG 3DB3 Databases - Provided knowledge on the operation and interaction with databases.
	\item SFWRENG 2XB3 Software Engineering Practice and Principles - Provided experience in software design, as well as providing knowledge in measuring system metrics.
\end{itemize}

\subsection{Knowledge/Skills Outside of Courses}

\plt{What skills/knowledge did you need to acquire for your capstone project
    that was outside of the courses you took?}

\end{document}
