
\documentclass{article}

\usepackage{tabularx}
\usepackage{booktabs}

\title{Reflection Report on \progname}

\author{\authname}

\date{}

%% Comments

\usepackage{color}

\newif\ifcomments\commentstrue %displays comments
%\newif\ifcomments\commentsfalse %so that comments do not display

\ifcomments
\newcommand{\authornote}[3]{\textcolor{#1}{[#3 ---#2]}}
\newcommand{\todo}[1]{\textcolor{red}{[TODO: #1]}}
\else
\newcommand{\authornote}[3]{}
\newcommand{\todo}[1]{}
\fi

\newcommand{\wss}[1]{\authornote{blue}{SS}{#1}} 
\newcommand{\plt}[1]{\authornote{magenta}{TPLT}{#1}} %For explanation of the template
\newcommand{\an}[1]{\authornote{cyan}{Author}{#1}}

%% Common Parts

\newcommand{\progname}{Software Engineering} % PUT YOUR PROGRAM NAME HERE
\newcommand{\authname}{Team 14, Reach
\\ Aamina Hussain
\\ David Moroniti
\\ Anika Peer
\\ Deep Raj
\\ Alan Scott} % AUTHOR NAMES                  

\usepackage{hyperref}
    \hypersetup{colorlinks=true, linkcolor=blue, citecolor=blue, filecolor=blue,
                urlcolor=blue, unicode=false}
    \urlstyle{same}
                                


\begin{document}

\maketitle

\plt{Reflection is an important component of getting the full benefits from a
    learning experience.  Besides the intrinsic benefits of reflection, this
    document will be used to help the TAs grade how well your team responded to
    feedback.  In addition, several CEAB (Canadian Engineering Accreditation Board)
    Learning Outcomes (LOs) will be assessed based on your reflections.}

\section{Changes in Response to Feedback}

\plt{Summarize the changes made over the course of the project in response to
    feedback from TAs, the instructor, teammates, other teams, the project
    supervisor (if present), and from user testers.}

\plt{For those teams with an external supervisor, please highlight how the feedback
    from the supervisor shaped your project.  In particular, you should highlight the
    supervisor's response to your Rev 0 demonstration to them.}

\plt{Version control can make the summary relatively easy, if you used issues
    and meaningful commits.  If you feedback is in an issue, and you responded in
    the issue tracker, you can point to the issue as part of explaining your
    changes.  If addressing the issue required changes to code or documentation, you
    can point to the specific commit that made the changes.}

\subsection{SRS and Hazard Analysis}

There were several changes made to the SRS and hazard analysis as a result of feedback from Dr. Smith, our TAs, peers, and supervisors.\\

SRS:

\begin{itemize}
    \item Peer feedback
          \begin{itemize}
              \item Update the individual product use cases to be fully black-box use cases. We did this by only focusing on what the use case is, and not describing how it could be done. This ensures the design is not constrained in the SRS (see \href{https://github.com/davimang/REACH/issues/12}{Issue 12}, \href{https://github.com/davimang/REACH/commit/7b80cb5d97721e50163f8874ba8e1d98c0bc45f9}{this commit} and \href{https://github.com/davimang/REACH/commit/916252afc158618ca8039c603c2f46ad836c0947}{this commit} for more details).
              \item Improve NFR-1 to make it less ambiguous (see \href{https://github.com/davimang/REACH/issues/14}{Issue 14} and \href{https://github.com/davimang/REACH/commit/6e514cf91ae0eae4c481530bf3403c99eb11770b}{this commit} for more details).
              \item Improve NFR-12 to make it less ambiguous (see \href{https://github.com/davimang/REACH/issues/15}{Issue 15} and \href{https://github.com/davimang/REACH/commit/6e514cf91ae0eae4c481530bf3403c99eb11770b}{this commit} for more details).
          \end{itemize}
    \item Dr. Smith and supervisors - After the revision 0 demo, both parties had very useful feedback, which we implemented in many places throughout the app and documentation.
          \begin{itemize}
              \item Add a map to be able to see exactly where the trial/study is being held (see \href{https://github.com/davimang/REACH/issues/138}{Issue 138} and \href{https://github.com/davimang/REACH/commit/8b6dd42a8c4585f0ac4a0efce1ea32024c65f6c0}{this commit} for more details). This was recommended by our supervisors.
              \item Add a disclaimer/title to the profile creation page to make the purpose of the page more clear (see \href{https://github.com/davimang/REACH/issues/155}{Issue 155} for more details). This was recommended by both Dr. Smith and our supervisors.
              \item Add the ability to filter the studies/trials by distance (see \href{https://github.com/davimang/REACH/issues/178}{Issue 178} and \href{https://github.com/davimang/REACH/commit/8b6dd42a8c4585f0ac4a0efce1ea32024c65f6c0}{this commit} for more details). This was recommended by both Dr. Smith and our supervisors.
          \end{itemize}
    \item Other
          \begin{itemize}
              \item Add FR-4, FR-5, FR-11 to the waiting room as they were not implemented.
          \end{itemize}
\end{itemize}

Some other useful issues to give more insight into the changes we have made to the SRS documentation are provided below:

\begin{itemize}
    \item \href{https://github.com/davimang/REACH/issues/251}{Issue 251}\\
\end{itemize}

Hazard Analysis:

\begin{itemize}
    \item Peer feedback/TA feedback
          \begin{itemize}
              \item Added some critical assumptions. This addresses feedback provided to us by our peers, and also feedback from the TA via the graded rubric (see \href{https://github.com/davimang/REACH/issues/35}{Issue 35} and \href{https://github.com/davimang/REACH/commit/643f128b7c4bd9c3072f56c813d5551e18c0af4e}{this commit} for more details).
              \item Added a definition of failure. This also addresses feedback provided to us by our peers and the TA (see \href{https://github.com/davimang/REACH/issues/36}{Issue 36} and \href{https://github.com/davimang/REACH/commit/3909aa4b28693cec5cc4ef3a9d7b84c41a6ffb24}{this commit} for more details).
              \item Add failure mode for email generation. This also addressed feedback provided to us by our peers and the TA (see \href{https://github.com/davimang/REACH/issues/36}{Issue 37} and \href{https://github.com/davimang/REACH/commit/e39f11a5c2fe7af0b6e261a353f818e2780ac43d}{this commit} for more details).
              \item Make SR-6 less ambiguous by improving the wording of the requirement. This addresses feedback provided to use by our peers (see \href{https://github.com/davimang/REACH/issues/39}{Issue 39} and \href{https://github.com/davimang/REACH/commit/bb0701657d9f217275ee0027e30f728f8924e277}{this commit} for more details).
          \end{itemize}
\end{itemize}

Some other useful issues to give more insight into the changes we have made to the Hazard Analysis documentation are provided below:

\begin{itemize}
    \item \href{https://github.com/davimang/REACH/issues/250}{Issue 250}
\end{itemize}


\subsection{Design and Design Documentation}

\subsection{VnV Plan and Report}

\section{Design Iteration (LO11)}

\plt{Explain how you arrived at your final design and implementation.  How did
    the design evolve from the first version to the final version?}

\section{Design Decisions (LO12)}

\plt{Reflect and justify your design decisions.  How did limitations,
    assumptions, and constraints influence your decisions?}

\section{Economic Considerations (LO23)}

\plt{Is there a market for your product? What would be involved in marketing your
    product? What is your estimate of the cost to produce a version that you could
    sell?  What would you charge for your product?  How many units would you have to
    sell to make money? If your product isn't something that would be sold, like an
    open source project, how would you go about attracting users?  How many potential
    users currently exist?}

\section{Reflection on Project Management (LO24)}

\plt{This question focuses on processes and tools used for project management.}

\subsection{How Does Your Project Management Compare to Your Development Plan}

\plt{Did you follow your Development plan, with respect to the team meeting plan,
    team communication plan, team member roles and workflow plan.  Did you use the
    technology you planned on using?}

\subsection{What Went Well?}

\plt{What went well for your project management in terms of processes and
    technology?}

\subsection{What Went Wrong?}

\plt{What went wrong in terms of processes and technology?}

\subsection{What Would you Do Differently Next Time?}

\plt{What will you do differently for your next project?}

\section{Reflection on Capstone}

\plt{This question focuses on what you learned during the course of the capstone project.}

\subsection{Which Courses Were Relevant}

\plt{Which of the courses you have taken were relevant for the capstone project?}

\subsection{Knowledge/Skills Outside of Courses}

\plt{What skills/knowledge did you need to acquire for your capstone project
    that was outside of the courses you took?}

\end{document}
