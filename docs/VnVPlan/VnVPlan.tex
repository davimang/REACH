\documentclass[12pt, titlepage]{article}

\usepackage{booktabs}
\usepackage{tabularx}
\usepackage{hyperref}
\usepackage{float}
\hypersetup{
    colorlinks,
    citecolor=blue,
    filecolor=black,
    linkcolor=red,
    urlcolor=blue
}
\usepackage[round]{natbib}

\input{../Comments}
%% Common Parts

\newcommand{\progname}{Software Engineering} % PUT YOUR PROGRAM NAME HERE
\newcommand{\authname}{Team 14, Reach
\\ Aamina Hussain
\\ David Moroniti
\\ Anika Peer
\\ Deep Raj
\\ Alan Scott} % AUTHOR NAMES                  

\usepackage{hyperref}
    \hypersetup{colorlinks=true, linkcolor=blue, citecolor=blue, filecolor=blue,
                urlcolor=blue, unicode=false}
    \urlstyle{same}
                                


\begin{document}

\title{Project Title: System Verification and Validation Plan for \progname{}} 
\author{\authname}
\date{\today}
	
\maketitle

\pagenumbering{roman}

\section*{Revision History}

\begin{tabularx}{\textwidth}{p{3cm}p{2cm}X}
\toprule {\bf Date} & {\bf Version} & {\bf Notes}\\
\midrule
Date 1 & 1.0 & Notes\\
Date 2 & 1.1 & Notes\\
\bottomrule
\end{tabularx}

~\\

\newpage

\tableofcontents

\listoftables

\newpage

\section{Symbols, Abbreviations, and Acronyms}

\renewcommand{\arraystretch}{1.2}
\begin{tabular}{l l} 
  \toprule		
  \textbf{symbol} & \textbf{description}\\
  \midrule 
  VnV & Verification and Validation\\
  V\&V & Verification and Validation\\
  \bottomrule
\end{tabular}\\

\newpage

\pagenumbering{arabic}

\section{General Information}

This document discusses the verification and validation plan for the REACH project. The main goal of this document
is to give all stakeholders in the REACH project confidence that the system will be built based on the correct 
requirements, and that those requirements will be implemented correctly, ensuring the system is functioning as intended.

\subsection{Summary}

The software being tested is REACH. REACH is a web application that will simplify the process for both patients, and clinicians working 
on behalf of their patients, of finding clinical trials that they are eligible for and that would be useful for them.

\subsection{Objectives}

The main objectives of this document are the following:

\begin{itemize}
  \item Ensure the system has a high level of usability
  \item Build confidence in the correctness of the searching/matching capabilities
  \item Build confidence in the security of user data
\end{itemize}

These three objectives are the most important qualities of the system, and they are the ones that are in scope for this project. \\

First, usability is a must
since the entire purpose of the application is to simplify the process of finding clinical trials. If the system was difficult to use, this process 
would not be simplified and the application would be useless. 
Furthermore, the ability to search/match patients to trials is an integral component of the application, and building confidence in 
that functionality is crucial for all stakeholders. Finally, ensuring the system is safe and secure when it comes to handling user data is of course 
extremely important for the users of the application which is why it is being prioritized.

\subsection{Relevant Documentation}

There are a few documents that are worth noting before jumping in to the verification and validation plan.
First, the SRS is the most important and relevant document that accompanies the plan  (\citet{SRS}). The majority of the test plan 
is built on top of the requirements specified in the SRS, since as stated previously, one of the main goals of this document is to 
build confidence in the implementation of the software requirements. Second, the development plan is relevant as it discusses various tools 
and technologies that can/will be used to test the software system being built {\citet{DevPlan}}. Finally, the MIS document which details the modular design 
of the system is especially relevant to the unit test portion of the document.

\section{Plan}

\wss{Introduce this section.   You can provide a roadmap of the sections to
  come.}

\subsection{Verification and Validation Team}

\wss{Your teammates.  Maybe your supervisor.
  You should do more than list names.  You should say what each person's role is
  for the project's verification.  A table is a good way to summarize this information.}

\begin{table}[htbp]
    % \centering
    \caption{ V\&V Team}
    \begin{tabular}{|c|c|p{10cm}|}
        \hline
        \textbf{Name} & \textbf{Team} & \textbf{Role} \\
        \hline 
        Aamina & Dev Team & As a front-end SME, Aamina will be responsible for handling functional and 
        nonfunctional testing for all front-end system capabilities. In particular, Aamina will manage 
        testing and verification pertaining to JS and TS frameworks, automated cloud deployment, graphical interface and presentation.\\
        \hline 
        David & Dev Team & As a back-end SME, David will be responsible for handling functional and nonfunctional testing 
        for all back-end system capabilities. In particular, David will manage testing and verification 
        pertaining to Python frameworks, database management, and API calls. \\
        \hline
        Anika & Dev Team & As a front-end SME, Anika will be responsible for handling functional and 
        nonfunctional testing for all front-end system capabilities. In particular, Anika will manage 
        testing and verification pertaining to JS and TS frameworks, automated cloud deployment, graphical interface and presentation.\\
        \hline
        Deep & Dev Team & As the testing SME, Deep will lead the majority of the testing efforts and ensure that the team is meeting
        the predetermined standards for testing and verification. Deep will focus on both
        front- and back-end testing and will lend his expertise to the rest of the dev team when they focus on end-to-end testing. \\
        \hline
        Alan & Dev Team & As a back-end SME, Alan will be responsible for handling functional and nonfunctional testing 
        for all back-end system capabilities. In particular, Alan will manage testing and verification 
        pertaining to Python frameworks, database management, and API calls. \\
        \hline
        All Developers & Dev Team & All developers will be responsible for writing and verifying unit, integration and 
        end-to-end tests for the system. Additionally, pipelines testing, Dockerfile testing, automated testing and any 
        other deployment-related testing is shared among all team-members.\\
        \hline
        Dr. Ho & Supervisor & As a supervisor, Dr.Ho will help the team verify the V\&V Plan and will ensure that 
        his vision is being tested properly at a high level.\\
        \hline
        Dr. Scallan & Supervisor & As a supervisor, Dr.Ho will help the team verify the V\&V Plan and will 
        ensure that his vision is being tested properly at a high level \\       
        \hline
    \end{tabular}
\end{table}

\subsection{SRS Verification Plan}

\wss{List any approaches you intend to use for SRS verification.  This may include
  ad hoc feedback from reviewers, like your classmates, or you may plan for 
  something more rigorous/systematic.}

\wss{Maybe create an SRS checklist?}

\subsection{Design Verification Plan}

\wss{Plans for design verification}

\wss{The review will include reviews by your classmates}

\wss{Create a checklists?}

\subsection{Verification and Validation Plan Verification Plan}

\wss{The verification and validation plan is an artifact that should also be
verified.  Techniques for this include review and mutation testing.}

\wss{The review will include reviews by your classmates}

\wss{Create a checklists?}

\subsection{Implementation Verification Plan}

To test the implementation first we have the tests listed below in section 4 and the unit tests listed in section 5. 
All of these tests should be run before each major update to the program (revision 0, revision 1, etc). Relevant tests should
also be run when the affected requirements are modified during minor updates. Also during minor updates the code should be 
inspected by at least one other developer using a github pull request. We will also be using static analysis, the specific 
software for this and other testing is listed in section 3.6 below.

\subsection{Automated Testing and Verification Tools}

For the backend (i.e., python service(s)), we will be using pytest as our unit testing framework. Additionally,
we will be using tavern (an API testing framework built on top of pytest) to test our API. These API tests will
ideally run at scheduled intervals (i.e., nightly, weekly... whatever is reasonable for the resources available), which can be 
done with the use of GitHub actions. Furthermore, we will be using several static analyzers to ensure code quality and maintainability. These 
static analyzers include flake8 (which ensures the code will follow the pep8 coding standard), mypy (which ensures strict typing in python,
and the use of comments+docstrings), and autoflake (which ensures there are no unused packages/imports in a python file). Finally, for analyzing 
code coverage in our python service(s), we will use coverage.py, which produces useful coverage reports after running tests using pytest.\\

For the frontend, nothing has changed from the development plan (i.e., using jest for unit testing + code coverage, and ESLint for linting).

\subsection{Software Validation Plan}

To validate our software system/requirements, there are a few steps that will be taken. First, continuing to meet with our clients (supervisors) while developing
the early parts of the application will help ensure the system is on the right track. Furthermore, upon completing of the revision 0 demo, we will
demo the system to our clients, enabling them to ask questions, provide feedback, and ensure the system has the correct requirements. At this
time we will also encourage our clients to use the platform directly (since it will be deployed, they will be able to easily access it) and give additional feedback post revision 0 demo,
which will hopefully bring light to new and improved requirements that can be implemented for revision 1. 

\section{System Test Description}
	
\subsection{Tests for Functional Requirements}

The following subsections are split based on the core aspects of the application. These include authentication, which enables a user 
to sign in to their account, user data, which enables users to save their data for future uses of the application, the emailing system, which will
notify users of new trials and give them a template for reaching out to trial clinicians, searching for trials (the main purpose of the app), and 
general program/system information (i.e., the FAQ).

\subsubsection{Authentication}

\begin{enumerate}

\item{AT-1\\}

Control: Automatic
					
Initial State: User is not signed in to the application, and is about to create an account.
					
Input: A valid, unique email, and a valid password satisfying the NIST guidelines
					
Output: The user's account should be created successfully, and a 201 CREATED response should be returned.

Test Case Derivation: As stated in FR-1 and FR-2, the system must only allow a user to create an account if the email is unique (i.e., not 
used by some other user yet), and if their password satisfies the NIST guidelines. Since the inputs to this test case satisfy these 
requirements, the response should be a success.
					
How test will be performed: Pre define a valid username and password. Automated test to pass in these values and check the return value/output.
					
\item{AT-2\\}

Control: Automatic
					
Initial State: User is not signed in to the application, and is about to create an account.
					
Input: An email that is already attached to another account, and a valid password.
					
Output: The user's account should not be created and a response stating this failure should be returned.

Test Case Derivation: From FR-2, the email must be unique. Since it is not unique in this test case, the account should not 
be created.

How test will be performed: Pre define an invalid email. Automated test to pass in these values and check the return value/output.

\item{AT-3\\}

Control: Automatic
					
Initial State: User is not signed in to the application, and is about to create an account.
					
Input: A valid, unique, email, and a password that does not satisfy the NIST guidelines.
					
Output: The user's account should not be created and a response stating this failure should be returned.

Test Case Derivation: From FR-1, the password must abide by the NIST guidelines. Since it does not in this test case, the account should not 
be created.

How test will be performed: Pre define an invalid password. Automated test to pass in these values and check the return value/output.


\item{AT-4\\}

Control: Automatic
					
Initial State: User is not signed in to the application, and is about to log in to the application.
					
Input: A valid, unique, email, and a password that satisfies the NIST guidelines, which are tied to an account.
					
Output: The user should be logged in to the application.

Test Case Derivation: From FR-3, if the credentials are correct, a user should be logged in to the application.

How test will be performed: Pre define a valid username and password tied to an account. Automated test to pass in these values and check the return value/output.


\item{AT-5\\}

Control: Automatic
					
Initial State: User is not signed in to the application, and is about to log in to the application.
					
Input: An email that is not tied to an account, with a password that satisfies NIST guidelines (but can be any valid password).
					
Output: The user should not be logged in to the application.

Test Case Derivation: From FR-3, a user is only logged in to an account if the credentials are correct. Since the email is not valid,
they should not be logged in.

How test will be performed: Pre define a random email/password. Automated test to pass in these values and check the return value/output.

\item{AT-6\\}

Control: Manual
					
Initial State: User is not signed in to the application, and would like to reset their password.
					
Input: A valid email address.
					
Output: The user should receive an email, containing a link which will allow them to reset their password.

Test Case Derivation: From FR-5, a user must receive an email when attempting to reset their password.

How test will be performed: Manually go on to the login page, enter an email address, and click reset password. Then, verify
that the email received is correct, and that the link takes the user to the reset password page.


\item{AT-7\\}

Control: Automatic
					
Initial State: User is not signed in to the application, and is about to reset their password.
					
Input: A password that is different from the one currently tied to their account, satisfying the NIST guidelines.
					
Output: The user's password should be reset, and they should be logged in to the application.

Test Case Derivation: From FR-5, a user must only be able to reset their password if the password they enter is new, and if it 
satisfies the NIST guidelines.

How test will be performed: Pre define a random email/password. Automated test to pass in these values and check the return value/output.

\item{AT-8\\}

Control: Manual
					
Initial State: User is signed in to the application and is about to sign out of the application.
					
Input: N/A
					
Output: The user should be logged out of the application.

Test Case Derivation: From FR-6, a user should be able to log out of the application, with really no constraints.

How test will be performed: Manually sign out of the application, and verify that it works.


\item{AT-9\\}

Control: Manual
					
Initial State: User has created an account for the first time, or is logging in as a guest.
					
Input: N/A
					
Output: The user is prompted to enter their personal data.

Test Case Derivation: From FR-7, a user is prompted to enter personal information when creating an account or logging in as a guest.

How test will be performed: Manually create an account and sign in as a guest, and verify the user is prompted.

\end{enumerate}

\subsubsection{User Data}

\begin{enumerate}

\item{DT-1\\}

Control: Manual
					
Initial State: User is not signed in, and does not have an account.
					
Input: New user data (medical conditions, personal attributes).
					
Output: Updated user database displayed on the user profile page.
	
Test Case Derivation: From FR-8, the system shall store the information entered by the user in the database.

How test will be performed: Manually create a new account, and add in some user data and medical conditions. Log back in and verify the data is still present on the user profile.


\item{DT-2\\}

Control: Manual
					
Initial State: User is signed in with their personal account.
					
Input: Updated user data (medical conditions, personal attributes).
					
Output: Updated user database displayed on the user profile page.

Test Case Derivation: From FR-9, a user shall be able to update their personal information.

How test will be performed: Manually update the user profile with new user information, then log out. Log back in and check that the new data is present.


\item{DT-3\\}

Control: Manual
					
Initial State: Tester is not logged in.
					
Input: Database query.
					
Output: List of non-unique unique database elements.

Test Case Derivation: From FR-8, the system shall store the information entered by the user in the database.

How test will be performed: Manually query the database for columns that are supposed to contain unique values, namely keys, such as user ID, email address, and etc. Verify that the length of the returned set is 0.

Rationale: To ensure that key constraints are upheld, we must ensure that the values are non-unique, otherwise querying for certain users may return multiple sets of data.


\end{enumerate}


\subsubsection{Email}

\begin{enumerate}


\item{ET-1\\}

Control: Manual
					
Initial State: The user has searched for trials, and has been given a list of trials.
					
Input: N/A
					
Output: A customized email template.

Test Case Derivation: From FR-10, the system shall create an email template for the user.

How test will be performed: Manually select a given trial result, and select the option to generate a template. Verify that it is displayed to the user.


\item{ET-1\\}

Control: Automatic
					
Initial State: The user is not logged in to the system.
					
Input: N/A
					
Output: Email notification notifying user of available trial.

Test Case Derivation: From FR-11, the system shall notify the user when new medical trials matching their specifications are available.

How test will be performed: A test user will be created with their own email account, with common medical conditions. The email will be checked periodically for notifications.


\end{enumerate}

\subsubsection{Program Information}

\begin{enumerate}

\item{PT-1\\}

Control: Manual
					
Initial State: The user is at the home screen.
					
Input: N/A
					
Output: FAQ menu.

Test Case Derivation: From FR-12, the system shall include frequently asked questions and trial information.

How test will be performed: Manually select the FAQ menu, and verify that it displays the FAQ.


\end{enumerate}

\subsubsection{Searching}

\begin{enumerate}

\item{ST-1\\}

Control: Manual
					
Initial State: The user is signed is, either as themselves or as a guest.
					
Input: Medical conditions and other user attributes used to search.
					
Output: Displayed search results.

Test Case Derivation: From FR-13, FR-14, a user shall be able to search for medical studies based on their health conditions, and the system shall display to the users the results of their trial search.

How test will be performed: Manually enter the a set of medical conditions and physical attributes into the search function. Submit, and verify the results are returned.


\item{ST-2\\}

Control: Manual
					
Initial State: The user is signed into their personal account, with medical information saved to their account.
					
Input: N/A
					
Output: Displayed search results.

Test Case Derivation: From FR-15, the system shall allow a signed-in user to automatically enter their conditions into the search parameters.

How test will be performed: Manually navigate to the search function. Verify that their conditions and information is automatically inputted into the search fields.


\item{ST-3\\}

Control: Manual
					
Initial State: The user is logged in, either as themselves or as a guest, and is at the search function.
					
Input: N/A
					
Output: N/A

Test Case Derivation: From FR-15, the system shall allow a signed-in user to automatically enter their conditions into the search parameters.

How test will be performed: Attempt to search without any fields filled in. Verify that the search fails.

\end{enumerate}

\subsection{Tests for Nonfunctional Requirements}

\subsubsection{Useability}

\begin{enumerate}

\item{UT-1}

Control: Dynamic, Manual
					
Initial State: The user starts on the home screen.

Test Case Derivation: From NFR-3, NFR-4, and NFR-5, regarding the learnability, simplicity, and legibility of the user interface.

How test will be performed: Have the user sign in, search, and browse results without outside help. 95\% of users should be able to accomplish this.


\item{UT-2}

Control: Dynamic, Manual
					
Initial State: The user starts on the home screen.

Test Case Derivation: From NFR-3, NFR-4, and NFR-5, regarding the learnability, simplicity, and legibility of the user interface.

How test will be performed: Have the user sign in, search, and browse results without outside help. 
Have the user then complete a survey(\nameref{sec:Appendix})

\item{UT-3}

Control: Dynamic, Manual
					
Initial State: The user starts on the home screen. The selected language should be set to the default (English).

Test Case Derivation: From NFR-17, regarding accessibility of the application users who do not speak an official language.

How test will be performed: Users of the next five most spoken languages other than English and French should be selected. They should be instructed to change to their preferred language and then should login, search, and browse results. 95\% of users should not report any issues navigating the interface.


\end{enumerate}

\subsubsection{Performance}
		
\begin{enumerate}

\item{PT-1\\}

Type: Dynamic, Manual
					
Initial State: User has entered their information for their and is about to search for eligible trials.
					
Input/Condition: All possible information filled in.
					
Output/Result: The system should load the trials and display them to the user in less than 5 seconds.
					
How test will be performed: In an existing account with all information categories filled in by the user, manually search for eligible trials, 
and keep track of how long it takes for the trials to load.
					
\item{PT-2\\}

Type: Dynamic, Manual
					
Initial State: N/A
					
Input/Condition: API key available to authenticate against the API.
					
Output/Result: API calls should respond in less than 1 second.
					
How test will be performed: Using locust, simulate 1000 users concurrently searching for trials, and manually track 
the max time recorded for a response from the API.

\end{enumerate}

\subsubsection{Operational and Environmental}
		
\begin{enumerate}

\item{OT-1\\}

Type: Automatic
					
Initial State: N/A
					
Input/Condition: Database is populated with running trials
					
Output/Result: The system should show the number of trials within an order of magnitude to the trials on ClinicalTrials.gov
					
How test will be performed: Count the number of running trials in the database and compare to the number of 
running trials on ClinicalTrials.gov

\item{OT-2\\}

Type: Dynamic, Manual
					
Initial State: User is on home screen, on a variety of devices
					
Output/Result: The user should report satisfactory performance on all devices (>7)
					
How test will be performed: Give the user multiple devices from the last 5 years (ex. iphone 15, windows laptop, chromebook, etc)
and ask them to rate the performance of the application compared to a desktop pc from 1-10.

\end{enumerate}

\subsubsection{Maintainability}

\begin{enumerate}

\item{MT-1\\}

Type: Static, Manual
					
How test will be performed: Manually run linters on code before merging a PR.

Rationale: This will ensure code is following the proper standards, which will help increase the maintainability of the system.
Also, this will be performed manually for the reason of conserving resources (i.e., could setup some workflow to automatically lint 
code every time some code is pushed to the GitHub Repo, however this could be costly).

\end{enumerate}

\subsubsection{Legal}

\begin{enumerate}
\item{LT-1}

Control: Dynamic, Manual
					
Initial State: The user is at the search screen.

Input: User search criteria.

Output: Option prompting user for consent.

Test Case Derivation: From NFR-18, regarding the collection of user consent before using or storing user data.

How test will be performed: Users are asked to input their search criteria, and should see a message prompting them to provide consent for the collection of data when they try to search.

\end{enumerate}

\subsection{Traceability Between Test Cases and Requirements}

\begin{table}[H]
  \caption{Test case Traceability}
  \begin{tabular}{|p{7cm}|p{7cm}|}
  \hline
  Test case & Requirement(s)\\
  \hline
  AT-1 & FR-1, FR-2\\
  \hline
  AT-2 & FR-2\\
  \hline
  AT-3 & FR-1\\
  \hline
  AT-4 & FR-3\\
  \hline
  AT-5 & FR-3\\
  \hline
  AT-6 & FR-5\\
  \hline
  AT-7 & FR-5\\
  \hline
  AT-8 & FR-6\\
  \hline
  AT-9 & FR-7\\
  \hline
  DT-1 & FR-8\\
  \hline
  DT-2 & FR-9\\
  \hline
  DT-3 & FR-8\\
  \hline
  ET-1 & FR-10\\
  \hline
  ET-2 & FR-11\\
  \hline
  PT-1 & FR-12\\
  \hline
  ST-1 & FR-13, FR-14\\
  \hline
  ST-2 & FR-15\\
  \hline
  ST-3 & FR-15\\
  \hline
  UT-1 & NFR-3, NFR-4, NFR-5\\
  \hline
  UT-2 & NFR-3, NFR-4, NFR-5\\
  \hline
  UT-3 & NFR-17\\
  \hline
  PT-1 & NFR-6\\
  \hline
  PT-2 & NFR-7\\
  \hline
  OT-1 & NFR-11\\
  \hline
  OT-2 & NFR-12\\
  \hline
 MT-1 & NFR-14\\
  \hline
 LT-1 & NFR-18\\
  \hline
  \end{tabular}
\end{table}

\section{Unit Test Description}

\wss{This section should not be filled in until after the MIS (detailed design
  document) has been completed.}

\wss{Reference your MIS (detailed design document) and explain your overall
philosophy for test case selection.}  

\wss{To save space and time, it may be an option to provide less detail in this section.  
For the unit tests you can potentially layout your testing strategy here.  That is, you 
can explain how tests will be selected for each module.  For instance, your test building 
approach could be test cases for each access program, including one test for normal behaviour 
and as many tests as needed for edge cases.  Rather than create the details of the input 
and output here, you could point to the unit testing code.  For this to work, you code 
needs to be well-documented, with meaningful names for all of the tests.}

\subsection{Unit Testing Scope}

\wss{What modules are outside of the scope.  If there are modules that are
  developed by someone else, then you would say here if you aren't planning on
  verifying them.  There may also be modules that are part of your software, but
  have a lower priority for verification than others.  If this is the case,
  explain your rationale for the ranking of module importance.}

\subsection{Tests for Functional Requirements}

\wss{Most of the verification will be through automated unit testing.  If
  appropriate specific modules can be verified by a non-testing based
  technique.  That can also be documented in this section.}

\subsubsection{Module 1}

\wss{Include a blurb here to explain why the subsections below cover the module.
  References to the MIS would be good.  You will want tests from a black box
  perspective and from a white box perspective.  Explain to the reader how the
  tests were selected.}

\begin{enumerate}

\item{test-id1\\}

Type: \wss{Functional, Dynamic, Manual, Automatic, Static etc. Most will
  be automatic}
					
Initial State: 
					
Input: 
					
Output: \wss{The expected result for the given inputs}

Test Case Derivation: \wss{Justify the expected value given in the Output field}

How test will be performed: 
					
\item{test-id2\\}

Type: \wss{Functional, Dynamic, Manual, Automatic, Static etc. Most will
  be automatic}
					
Initial State: 
					
Input: 
					
Output: \wss{The expected result for the given inputs}

Test Case Derivation: \wss{Justify the expected value given in the Output field}

How test will be performed: 

\item{...\\}
    
\end{enumerate}

\subsubsection{Module 2}

...

\subsection{Tests for Nonfunctional Requirements}

\wss{If there is a module that needs to be independently assessed for
  performance, those test cases can go here.  In some projects, planning for
  nonfunctional tests of units will not be that relevant.}

\wss{These tests may involve collecting performance data from previously
  mentioned functional tests.}

\subsubsection{Module ?}
		
\begin{enumerate}

\item{test-id1\\}

Type: \wss{Functional, Dynamic, Manual, Automatic, Static etc. Most will
  be automatic}
					
Initial State: 
					
Input/Condition: 
					
Output/Result: 
					
How test will be performed: 
					
\item{test-id2\\}

Type: Functional, Dynamic, Manual, Static etc.
					
Initial State: 
					
Input: 
					
Output: 
					
How test will be performed: 

\end{enumerate}

\subsubsection{Module ?}

...

\subsection{Traceability Between Test Cases and Modules}

\wss{Provide evidence that all of the modules have been considered.}
				
\bibliographystyle{plainnat}

\bibliography{../../refs/References}

\newpage

\section{Appendix}

This is where you can place additional information.

\subsection{Symbolic Parameters}

The definition of the test cases will call for SYMBOLIC\_CONSTANTS.
Their values are defined in this section for easy maintenance.

\subsection{Usability Survey Questions}
\label{sec:Appendix}
Usability Survey:

How easy was it to navigate to the search page (1 is hard, 10 is easy)?

How easy was it to understand each search term (1 is hard, 10 is easy)?

Did you find any part of the process frustrating(Y/N)?

If yes, which part did you find frustrating and why?

Was all text easy to read while using the website(Y/N)?


\newpage{}
\section*{Appendix --- Reflection}

Below are skills that the team will need to develop in order to successfully carry out the verification and validation process, in addition to how the team plans to further our knowledge in the given areas.

\begin{itemize}
\item Automated unit testing via integrations
\begin{itemize}
\item Free tutorials on Youtube or other free video sharing platforms.
\item Documentation on the given cloud platform's integrations.
\end{itemize}
\item Verification and validation reporting standards
\begin{itemize}
\item Lecture material and other course material, from both capstone and other courses.
\item Documentation and templates from IEEE.
\end{itemize}
\end{itemize}
For the automated unit testing, the team has decided to pursue video tutorials, as these tend to be easier to approach, and provide practical examples. For the VnV reporting, the team has decided to use course materials, as this is readily accessible and easy to understand.


Additionally, below is a skill that each member will need to develop, along with two ways they can learn each skill.
\begin{itemize}
\item Alan - GitHub CI/CD pipelines
\begin{itemize}
\item YouTube tutorials or video tutorials from other sites.
\item GitHub documentation.
\end{itemize}
\item David - Jest testing framework
\begin{itemize}
  \item Jest documentation
  \item YouTube videos/tutorials 
\end{itemize}
\end{itemize}

Below is which method each member intends to use, and why.
\begin{itemize}
\item Alan - YouTube tutorials. This method is the most approachable, and provides easy to follow along examples.
\item David - Jest documentation. I will use the documentation since I have used several other testing frameworks in previous projects, and I anticipate the documentation should provide sufficient information to get up and running with the framework.
\end{itemize}


\end{document}