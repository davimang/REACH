\documentclass{article}

\usepackage{booktabs}
\usepackage{tabularx}
\usepackage{hyperref}
\usepackage{float}
\usepackage{pdflscape}
\usepackage{multirow}
\usepackage{geometry}

\hypersetup{
    colorlinks=true,       % false: boxed links; true: colored links
    linkcolor=red,          % color of internal links (change box color with linkbordercolor)
    citecolor=green,        % color of links to bibliography
    filecolor=magenta,      % color of file links
    urlcolor=cyan           % color of external links
}

\geometry{margin=1.25in}

\title{Hazard Analysis\\\progname}

\author{\authname}

\date{}

%% Comments

\usepackage{color}

\newif\ifcomments\commentstrue %displays comments
%\newif\ifcomments\commentsfalse %so that comments do not display

\ifcomments
\newcommand{\authornote}[3]{\textcolor{#1}{[#3 ---#2]}}
\newcommand{\todo}[1]{\textcolor{red}{[TODO: #1]}}
\else
\newcommand{\authornote}[3]{}
\newcommand{\todo}[1]{}
\fi

\newcommand{\wss}[1]{\authornote{blue}{SS}{#1}} 
\newcommand{\plt}[1]{\authornote{magenta}{TPLT}{#1}} %For explanation of the template
\newcommand{\an}[1]{\authornote{cyan}{Author}{#1}}

%% Common Parts

\newcommand{\progname}{Software Engineering} % PUT YOUR PROGRAM NAME HERE
\newcommand{\authname}{Team 14, Reach
\\ Aamina Hussain
\\ David Moroniti
\\ Anika Peer
\\ Deep Raj
\\ Alan Scott} % AUTHOR NAMES                  

\usepackage{hyperref}
    \hypersetup{colorlinks=true, linkcolor=blue, citecolor=blue, filecolor=blue,
                urlcolor=blue, unicode=false}
    \urlstyle{same}
                                


\begin{document}

\maketitle
\thispagestyle{empty}

~\newpage

\pagenumbering{roman}

\begin{table}[hp]
\caption{Revision History} \label{TblRevisionHistory}
\begin{tabularx}{\textwidth}{llX}
\toprule
\textbf{Date} & \textbf{Developer(s)} & \textbf{Change}\\
\midrule
Date1 & Name(s) & Description of changes\\
October 18, 2023 & Aamina Hussain & Added sections 1, 2, and 4\\
... & ... & ...\\
\bottomrule
\end{tabularx}
\end{table}

~\newpage

\tableofcontents

~\newpage

\pagenumbering{arabic}

\wss{You are free to modify this template.}

\section{Introduction}
This document includes a hazard analysis for the web application REACH. REACH will allow users 
to find clinical trials or research studies they are eligible to participate in. It pulls in information about these studies 
from existing external public databases. This document will analyze and record any hazards to the system REACH. In this case, 
a hazard is a property of a system, together with the condition of the environment the system is in, which can cause harm or 
damage and results in a loss. This definition of hazard is from Nancy Leveson's work.

\wss{You can include your definition of what a hazard is here.}

\section{Scope and Purpose of Hazard Analysis}
The scope and purpose of this hazard analysis is to identify any system hazards and which components they are related to. This 
includes analyzing the causes and effects of the hazard and the recommended actions to mitigate the hazard, as well as documenting 
the resulting safety and security requirements.

\section{System Boundaries and Components}

\section{Critical Assumptions}
N/A. There are no critical assumptions being made about the software or system.

\wss{These assumptions that are made about the software or system.  You should
minimize the number of assumptions that remove potential hazards.  For instance,
you could assume a part will never fail, but it is generally better to include
this potential failure mode.}

\section{Failure Mode and Effect Analysis}


    
\begin{table}[htbp]
    \small
    \centering
    \resizebox{\columnwidth}{!}{\begin{tabular}{|p{3cm}|p{3cm}p{3cm}p{3cm}p{3cm}p{3cm}p{3cm}|}
    \hline
    Component & Failure modes & Effects & Causes & Action & SR & Ref. \\
    \hline
    \multirow{3}{=}{Trial Fetching/Matching} & {External Api's unavailable} & {System is unable to search for trials} & {System failure on the API providers side, scheduled maintenance, API access method changed} & {Make application unavailable during scheduled maintenance, prevent users from searching if API is down} & {Place Holder} & {Place Holder} \\\cline{2-7}
    & {Mismatch in trials being recommended} & {User attempts to sign up for ineligible trial} & {} & {Display a warning/disclaimer with respect to signing up for trials} & {Place Holder} & {Place Holder} \\\cline{2-7}
    & {User eligible for "too many" trials} & {Too many emails being sent to user and it could make it more difficult for a user to find a trial they really like.} & {Not enough data entered by user.} & {Inform user if they haven't entered enough data to get a good search.} & {Place Holder} & {Place Holder}
    \\\hline
    Place holder & Place holder & Place holder & Place holder & Place holder & Place holder & Place holder \\
    \hline
    Place holder & Place holder & Place holder & Place holder & Place holder & Place holder & Place holder \\
    \hline
    \end{tabular}}
\end{table}


\section{Safety and Security Requirements}

\wss{Newly discovered requirements.  These should also be added to the SRS.  (A
rationale design process how and why to fake it.)}

\section{Roadmap}

\wss{Which safety requirements will be implemented as part of the capstone timeline?
Which requirements will be implemented in the future?}

\end{document}