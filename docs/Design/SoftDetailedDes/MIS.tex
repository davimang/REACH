\documentclass[12pt, titlepage]{article}

\usepackage{amsmath, mathtools}

\usepackage[round]{natbib}
\usepackage{amsfonts}
\usepackage{amssymb}
\usepackage{graphicx}
\usepackage{colortbl}
\usepackage{xr}
\usepackage{hyperref}
\usepackage{longtable}
\usepackage{xfrac}
\usepackage{tabularx}
\usepackage{float}
\usepackage{siunitx}
\usepackage{booktabs}
\usepackage{multirow}
\usepackage[section]{placeins}
\usepackage{caption}
\usepackage{fullpage}

\hypersetup{
bookmarks=true,     % show bookmarks bar?
colorlinks=true,       % false: boxed links; true: colored links
linkcolor=red,          % color of internal links (change box color with linkbordercolor)
citecolor=blue,      % color of links to bibliography
filecolor=magenta,  % color of file links
urlcolor=cyan          % color of external links
}

\usepackage{array}

\externaldocument{../../SRS/SRS}

%% Comments

\usepackage{color}

\newif\ifcomments\commentstrue %displays comments
%\newif\ifcomments\commentsfalse %so that comments do not display

\ifcomments
\newcommand{\authornote}[3]{\textcolor{#1}{[#3 ---#2]}}
\newcommand{\todo}[1]{\textcolor{red}{[TODO: #1]}}
\else
\newcommand{\authornote}[3]{}
\newcommand{\todo}[1]{}
\fi

\newcommand{\wss}[1]{\authornote{blue}{SS}{#1}} 
\newcommand{\plt}[1]{\authornote{magenta}{TPLT}{#1}} %For explanation of the template
\newcommand{\an}[1]{\authornote{cyan}{Author}{#1}}

%% Common Parts

\newcommand{\progname}{Software Engineering} % PUT YOUR PROGRAM NAME HERE
\newcommand{\authname}{Team 14, Reach
\\ Aamina Hussain
\\ David Moroniti
\\ Anika Peer
\\ Deep Raj
\\ Alan Scott} % AUTHOR NAMES                  

\usepackage{hyperref}
    \hypersetup{colorlinks=true, linkcolor=blue, citecolor=blue, filecolor=blue,
                urlcolor=blue, unicode=false}
    \urlstyle{same}
                                


\begin{document}

\title{Module Interface Specification for \progname{}}

\author{\authname}

\date{\today}

\maketitle

\pagenumbering{roman}

\section{Revision History}

\begin{tabularx}{\textwidth}{p{3cm}p{2cm}X}
\toprule {\bf Date} & {\bf Version} & {\bf Notes}\\
\midrule
Date 1 & 1.0 & Notes\\
Date 2 & 1.1 & Notes\\
\bottomrule
\end{tabularx}

~\newpage

\section{Symbols, Abbreviations and Acronyms}

See SRS Documentation at \wss{give url}

\wss{Also add any additional symbols, abbreviations or acronyms}

\newpage

\tableofcontents

\newpage

\pagenumbering{arabic}

\section{Introduction}

The following document details the Module Interface Specifications for
REACH, a web application used to improve patients' access to clinical 
trials and practitioners' access to potential participants. More specifically,
it will provide the list of modules that have been decomposed from the Module Guide,
each with their interface specification, detailing important characteristics such as 
the module's methods and state variables.

Complementary documents, such as the System Requirement Specifications
and Module Guide can be found at \url{https://github.com/davimang/REACH}.

\section{Notation}

The structure of the MIS for modules comes from \citet{HoffmanAndStrooper1995},
with the addition that template modules have been adapted from
\cite{GhezziEtAl2003}.  The mathematical notation comes from Chapter 3 of
\citet{HoffmanAndStrooper1995}.  For instance, the symbol := is used for a
multiple assignment statement and conditional rules follow the form $(c_1
\Rightarrow r_1 | c_2 \Rightarrow r_2 | ... | c_n \Rightarrow r_n )$.

The following table summarizes the primitive data types used by \progname. 

\begin{center}
\renewcommand{\arraystretch}{1.2}
\noindent 
\begin{tabular}{l l p{7.5cm}} 
\toprule 
\textbf{Data Type} & \textbf{Notation} & \textbf{Description}\\ 
\midrule
character & char & a single symbol or digit\\
integer & $\mathbb{Z}$ & a number without a fractional component in (-$\infty$, $\infty$) \\
natural number & $\mathbb{N}$ & a number without a fractional component in [1, $\infty$) \\
real & $\mathbb{R}$ & any number in (-$\infty$, $\infty$)\\
\bottomrule
\end{tabular} 
\end{center}

\noindent
The specification of \progname \ uses some derived data types: sequences, strings, and
tuples. Sequences are lists filled with elements of the same data type. Strings
are sequences of characters. Tuples contain a list of values, potentially of
different types. In addition, \progname \ uses functions, which
are defined by the data types of their inputs and outputs. Local functions are
described by giving their type signature followed by their specification.

\section{Module Decomposition}

The following table is taken directly from the Module Guide document for this project.

\begin{table}[h!]
\centering
\begin{tabular}{p{0.3\textwidth} p{0.6\textwidth}}
\toprule
\textbf{Level 1} & \textbf{Level 2}\\
\midrule

{Hardware-Hiding} & ~ \\
\midrule

\multirow{7}{0.3\textwidth}{Behaviour-Hiding} & Input Parameters\\
& Output Format\\
& Output Verification\\
& Temperature ODEs\\
& Energy Equations\\ 
& Control Module\\
& Specification Parameters Module\\
\midrule

\multirow{3}{0.3\textwidth}{Software Decision} & {Sequence Data Structure}\\
& ODE Solver\\
& Plotting\\
\bottomrule

\end{tabular}
\caption{Module Hierarchy}
\label{TblMH}
\end{table}

\newpage
~\newpage


%%%%%%%%%%%%%%% User data %%%%%%%%%%%%%%

\section{MIS of the User data module} \label{User}

\subsection{Module}

User

\subsection{Uses}
PatientInfo, Trial

\subsection{Syntax}

\subsubsection{Exported Constants}
None

\subsubsection{Exported Access Programs}

\begin{center}
\begin{tabular}{p{4cm} p{4cm} p{4cm} p{4cm}}
\hline
\textbf{Name} & \textbf{In} & \textbf{Out} & \textbf{Exceptions} \\
\hline
getName & - & seq of char & - \\
\hline
setName & seq of char & - & EmptyName \\
\hline
getEmail & - & seq of char & - \\
\hline
setEmail & seq of char & - & InvalidEmail \\
\hline
getInfoProfiles  & - & list of InfoProfile & - \\
\hline
getInfoProfile & integer & InfoProfile & InvalidInfoProfileId \\
\hline
addInfoProfile & InfoProfile & - & - \\
\hline
removeInfoProfile & integer & - & - \\
\hline
addTrial & Trial & - & - \\
\hline
removeTrial & integer & - & - \\
\hline
getTrials & - & list of Trial & - \\
\hline
getTrial & integer & Trial & InvalidTrialId \\
\end{tabular}
\end{center}

\subsection{Semantics}

\subsubsection{State Variables}

name: seq of char\\
email: seq of char\\
infoProfiles: list of PatientInfo\\
trials: list of Trial

\subsubsection{Environment Variables}
None

\subsubsection{Assumptions}
\begin{itemize}
  \item Each InfoProfile has a unique id.
  \item Each Trial has a unique id.
\end{itemize}

\subsubsection{Access Routine Semantics}

\noindent getName():
\begin{itemize}
\item transition: N/A
\item output: out := self.name
\item exception: N/A
\end{itemize}

\noindent setName(newName: seq of char):
\begin{itemize}
\item transition: self.name := newName
\item output: N/A
\item exception: $exc := length(newName) == 0 \Rightarrow EmptyName$
\end{itemize}

\noindent getEmail():
\begin{itemize}
\item transition: N/A
\item output: out := self.email
\item exception: N/A
\end{itemize}

\noindent setEmail(newEmail: seq of char):
\begin{itemize}
\item transition: self.email := newEmail
\item output: N/A
\item exception: $exc := isInvalidEmail(newEmail) \Rightarrow InvalidEmail$
\end{itemize}

\noindent getInfoProfiles():
\begin{itemize}
\item transition: N/A
\item output: out := self.infoProfiles
\item exception: N/A
\end{itemize}

\noindent getInfoProfile(id: integer):
\begin{itemize}
\item transition: N/A
\item output: out := $\{\exists i \in self.infoProfiles | i.id = id\} \Rightarrow i$
\item exception: exc := $\neg\{\exists i \in self.infoProfiles | i.id = id\} \Rightarrow InvalidInfoProfileId$
\end{itemize}

\noindent addInfoProfile(newInfoProfile: InfoProfile):
\begin{itemize}
\item transition: self.infoProfiles = self.infoProfiles + newInfoProfile (add the new infoprofile to the list of info profiles connected to the 
current user)
\item output: N/A
\item exception: N/A
\end{itemize}

\noindent removeInfoProfile(oldInfoProfile: InfoProfile):
\begin{itemize}
\item transition: self.infoProfiles = self.infoProfiles - oldInfoProfile (remove the info profile passed to the method from the list of info 
profiles connected to the current user)
\item output: N/A
\item exception: N/A
\end{itemize}

\noindent getTrials():
\begin{itemize}
\item transition: N/A
\item output: out := self.trials
\item exception: N/A
\end{itemize}

\noindent getTrial(id: integer):
\begin{itemize}
\item transition: N/A
\item output: out := $\{\exists i \in self.trials | i.id = id\} \Rightarrow i$
\item exception: exc := $\neg\{\exists i \in self.trials | i.id = id\} \Rightarrow InvalidTrialId$
\end{itemize}

\noindent addTrial(newTrial: Trial):
\begin{itemize}
\item transition: self.trials = self.trials + newTrial (same idea as add info profiles)
\item output: N/A
\item exception: N/A
\end{itemize}

\noindent removeTrial(oldTrial: Trial):
\begin{itemize}
\item transition: self.trials = self.trials - oldTrial (same idea as remove info profiles)
\item output: N/A
\item exception: N/A
\end{itemize}

\subsubsection{Local Functions}
\noindent saveUser(user: User):
\begin{itemize}
\item transition: Saves the user to the database.
\item output: N/A
\item exception: N/A
\end{itemize}

\noindent loadUser(user: User):
\begin{itemize}
\item transition: Loads the user from the database.
\item output: N/A
\item exception: N/A
\end{itemize}


%%%%%%%%%%%%%%% Info Profiles %%%%%%%%%%%%%%
\section{MIS of Info Profile data module} \label{InfoProfile}

\subsection{Module}

InfoProfile

\subsection{Uses}
None

\subsection{Syntax}

\subsubsection{Exported Constants}
None

\subsubsection{Exported Access Programs}

\begin{center}
\begin{tabular}{p{4cm} p{4cm} p{4cm} p{4cm}}
\hline
\textbf{Name} & \textbf{In} & \textbf{Out} & \textbf{Exceptions} \\
\hline
getDOB & - & datetime & - \\
\hline
setDOB & datetime & - & - \\
\hline
getAddress & - & seq of char & - \\
\hline
setAddress & seq of char & - & - \\
\hline
getGender & - & seq of char & - \\
\hline
setGender & seq of char & - & - \\
\hline
getHealthDetails & - & map$<$seq of char: Any$>$ & - \\
\hline
setHealthDetails & map$<$seq of char: Any$>$ & - & - \\

\end{tabular}
\end{center}

\subsection{Semantics}

\subsubsection{State Variables}

dateOfBirth: datetime\\
address: seq of char\\
gender: seq of char\\
healthDetails: map$<$seq of char: Any$>$

\subsubsection{Environment Variables}
None

\subsubsection{Assumptions}
None

\subsubsection{Access Routine Semantics}

\noindent getDOB():
\begin{itemize}
\item transition: N/A
\item output: out := self.dateOfBirth
\item exception: N/A
\end{itemize}

\noindent setDOB(newDOB: datetime):
\begin{itemize}
\item transition: self.dateOfBirth = newDOB
\item output: N/A
\item exception: N/A
\end{itemize}

\noindent getAddress():
\begin{itemize}
\item transition: N/A
\item output: out := self.address
\item exception: N/A
\end{itemize}

\noindent setAddress(newAddress: seqOfChar):
\begin{itemize}
\item transition: self.address = newAddress
\item output: N/A
\item exception: N/A
\end{itemize}

\noindent getGender():
\begin{itemize}
\item transition: N/A
\item output: out := self.gender
\item exception: N/A
\end{itemize}

\noindent setGender(gender: seq of char):
\begin{itemize}
\item transition: self.gender = gender
\item output: N/A
\item exception: N/A
\end{itemize}

\noindent getHealthDetails():
\begin{itemize}
\item transition: N/A
\item output: out := self.healthDetails
\item exception: N/A
\end{itemize}

\noindent setHealthDetails(newHealthDetails: map$<$seq of char: Any$>$):
\begin{itemize}
\item transition: self.healthDetails = newHealthDetails
\item output: N/A
\item exception: N/A
\end{itemize}

\subsubsection{Local Functions}
\noindent saveInfoProfile(infoProfile: InfoProfile):
\begin{itemize}
\item transition: Saves the info profile to the database.
\item output: N/A
\item exception: N/A
\end{itemize}

\noindent loadInfoProfile(infoProfile: InfoProfile):
\begin{itemize}
\item transition: Loads the info profile from the database.
\item output: N/A
\item exception: N/A
\end{itemize}

\newpage

%%%%%%%%%%%%%%% Saved Trials %%%%%%%%%%%%%%

\section{MIS of Trial data module} \label{Trial}

\subsection{Module}
Trial

\subsection{Uses}
None

\subsection{Syntax}

\subsubsection{Exported Constants}
None

\subsubsection{Exported Access Programs}

\begin{center}
\begin{tabular}{p{4cm} p{4cm} p{4cm} p{4cm}}
\hline
\textbf{Name} & \textbf{In} & \textbf{Out} & \textbf{Exceptions} \\
\hline
getTitle & - & seq of char & - \\
\hline
getDescription & - & seq of char & - \\
\hline
getUrl & - & seq of char & - \\
\end{tabular}
\end{center}

\subsection{Semantics}

\subsubsection{State Variables}
title: seq of char\\
description: seq of char\\
url: seq of char

\subsubsection{Environment Variables}
None

\subsubsection{Assumptions}

\begin{itemize}
  \item Trial details (title, description, url) will not need to be changed. Therefore no need for setter methods to update these values for a certain trial.
\end{itemize}

\subsubsection{Access Routine Semantics}

\noindent getTitle():
\begin{itemize}
\item transition: N/A
\item output: out := self.title
\item exception: N/A
\end{itemize}

\noindent getDescription():
\begin{itemize}
\item transition: N/A
\item output: out := self.description
\item exception: N/A
\end{itemize}

\noindent getUrl():
\begin{itemize}
\item transition: N/A
\item output: out := self.url
\item exception: N/A
\end{itemize}

\subsubsection{Local Functions}
\noindent saveTrial(trial: Trial):
\begin{itemize}
\item transition: Saves the trial to the database.
\item output: N/A
\item exception: N/A
\end{itemize}

\noindent loadTrial(trial: Trial):
\begin{itemize}
\item transition: Loads the trial from the database.
\item output: N/A
\item exception: N/A
\end{itemize}

\newpage


%%%%%%%%%%%%%%% Fetch Trials %%%%%%%%%%%%%%

\section{MIS of the Fetch Trials Modules} \label{TrialFetcher}

\subsection{Module}

TrialFetcher

\subsection{Uses}

Trial

\subsection{Syntax}

\subsubsection{Exported Constants}
None

\subsubsection{Exported Access Programs}

\begin{center}
\begin{tabular}{p{4cm} p{4cm} p{4cm} p{4cm}}
\hline
\textbf{Name} & \textbf{In} & \textbf{Out} & \textbf{Exceptions} \\
\hline
getTrials & seq. of String, integer, String & DataFrame & MissingParameter, InvalidAge, InvalidAddress \\
\hline
getLocator & - & geocoder & - \\
\hline
setLocator & geocoder & - & - \\
\hline
\end{tabular}
\end{center}

\subsection{Semantics}

\subsubsection{State Variables}
locator: geocoder \\

\subsubsection{Environment Variables}
None

\subsubsection{Assumptions}
\begin{itemize}
  \item Each Trial has a unique id.
  \item The trial API will always be accessible.
\end{itemize}

\subsubsection{Access Routine Semantics}

\noindent getTrials(conditions: sequence of String, age: int, address: String):
\begin{itemize}
\item transition: None
\item output: out := DataFrame populated with trials from the ClinicalTrials.gov API
\item exception: $exc := (age \notin (0,120] \rightarrow InvalidAge) \lor (\neg checkAddress(address) \rightarrow InvalidAddress) \lor ((\exists x . x \in parameters : x = \varepsilon) \rightarrow MissingParameter)$
\end{itemize}

\subsubsection{Local Functions}
\noindent convertTrialsToDataFrame(rawData: csv):
\begin{itemize}
\item transition: None
\item output: out := rawData formatted as a DataFrame
\item exception: None
\end{itemize}

\noindent checkAddress(address: String):
\begin{itemize}
\item transition: None
\item output: out:= True
\item exception: $exc := (geopy.geolocator(address) = exception \rightarrow False)$
\end{itemize}

%%%%%%%%%%%%%%% Filter Trials %%%%%%%%%%%%%%

\section{MIS of the Trial Filtering Module} \label{TrialFilterer}

\subsection{Module}

TrialFilterer

\subsection{Uses}

Trial

\subsection{Syntax}

\subsubsection{Exported Constants}
None

\subsubsection{Exported Access Programs}

\begin{center}
\begin{tabular}{p{4cm} p{4cm} p{4cm} p{4cm}}
\hline
\textbf{Name} & \textbf{In} & \textbf{Out} & \textbf{Exceptions} \\
\hline
exportTrials & - & json & - \\
\hline
fetchTrials & seq. of String, integer, String & - & - \\
\hline
getLocator & - & geocoder & - \\
\hline
setLocator & geocoder & - & - \\
\hline
\end{tabular}
\end{center}

\subsection{Semantics}

\subsubsection{State Variables}
locator: geocoder \\
trials: DataFrame

\subsubsection{Environment Variables}
None

\subsubsection{Assumptions}
\begin{itemize}
  \item Each Trial has a unique id.
  \item Exceptions are caught downstream by the TrialFetcher module
\end{itemize}

\subsubsection{Access Routine Semantics}

\noindent fetchTrials(conditions: sequence of String, age: int, address: String):
\begin{itemize}
\item transition: self.trials populated with trials via TrialFilterer module
\item output: None
\item exception: None
\end{itemize}

\noindent exportTrials():
\begin{itemize}
\item transition: None
\item output: out:= self.trials as json
\item exception: None
\end{itemize}

\noindent getLocator():
\begin{itemize}
\item transition: None
\item output: out:= self.locator
\item exception: None
\end{itemize}

\noindent setLocator(loc: geocoder):
\begin{itemize}
\item transition: self.locator = loc
\item output: None
\item exception: None
\end{itemize}

\subsubsection{Local Functions}
\noindent cleanAge(stringAge: String):
\begin{itemize}
\item transition: None
\item output: out:= $(inMonths \rightarrow int(stringAge)/12) \rightarrow int(stringAge)$
\item exception:  $exc := stringAge \notin \mathbb{R} \rightarrow InvalidAge$
\end{itemize}

\noindent geodesicDistance(address: geocode, trialLocation: geocode):
\begin{itemize}
\item transition: None
\item output: $out := \arccos(\sin(address.latitude)\cdot\sin(trialLocation.latitude) + \cos(address.latitude)\cdot\cos(ltrialLocation.latitude)\cdot\cos(trialLocation.longitude-address.longitude) ) \cdot 6371000$
\item exception: None
\end{itemize}

\noindent calculateDistance():
\begin{itemize}
\item transition: $self.trials[distance] \mapsto geodesicDistance(address, self.trials[trialLocation])$
\item output: None
\item exception: None
\end{itemize}

\noindent convertToJSON(df: DataFrame):
\begin{itemize}
\item transition: None
\item output:  $df \rightarrow json(df)$
\item exception: None
\end{itemize}

\section{MIS of the Registration Module} \label{Registration}

\subsection{Module}

Registration

\subsection{Uses}

User

\subsection{Syntax}

\subsubsection{Exported Constants}
None

\subsubsection{Exported Access Programs}

\begin{center}
\begin{tabular}{p{4cm} p{4cm} p{4cm} p{4cm}}
\hline
\textbf{Name} & \textbf{In} & \textbf{Out} & \textbf{Exceptions} \\
\hline
registerUser & String, String & Boolean & - \\
\hline
\end{tabular}
\end{center}

\subsection{Semantics}

\subsubsection{State Variables}

\subsubsection{Environment Variables}

\subsubsection{Assumptions}

None

\subsubsection{Access Routine Semantics}

\noindent registerUser(emailAddress: String, password: String):
\begin{itemize}
\item transition: None
\item output: out := True if the Registration was successful, False otherwise
\item exception: None
\end{itemize}

\subsubsection{Local Functions}

\section{MIS of the Login Module} \label{Login}

\subsection{Module}

Login

\subsection{Uses}

User

\subsection{Syntax}

\subsubsection{Exported Constants}
None

\subsubsection{Exported Access Programs}

\begin{center}
\begin{tabular}{p{4cm} p{4cm} p{4cm} p{4cm}}
\hline
\textbf{Name} & \textbf{In} & \textbf{Out} & \textbf{Exceptions} \\
\hline
loginUser & String, String & Boolean & - \\
\hline
\end{tabular}
\end{center}

\subsection{Semantics}

\subsubsection{State Variables}

\subsubsection{Environment Variables}

\subsubsection{Assumptions}

None

\subsubsection{Access Routine Semantics}

\noindent loginUser(emailAddress: String, password: String):
\begin{itemize}
\item transition: None
\item output: out := True if the login was successful, False otherwise
\item exception: None
\end{itemize}

\subsubsection{Local Functions}

\section{MIS of the Email Template Module} \label{EmailTemplate}

\subsection{Module}

EmailTemplate

\subsection{Uses}

User, Trial

\subsection{Syntax}

\subsubsection{Exported Constants}
None

\subsubsection{Exported Access Programs}

\begin{center}
\begin{tabular}{p{4cm} p{4cm} p{4cm} p{4cm}}
\hline
\textbf{Name} & \textbf{In} & \textbf{Out} & \textbf{Exceptions} \\
\hline
createEmail & User, Trial & String & - \\
\hline
\end{tabular}
\end{center}

\subsection{Semantics}

\subsubsection{State Variables}

\subsubsection{Environment Variables}

\subsubsection{Assumptions}

\subsubsection{Access Routine Semantics}

\noindent createEmail(user: User, trial: Trial):
\begin{itemize}
\item transition: None
\item output: out := email template personalized using User and Trial data
\item exception: None
\end{itemize}

\subsubsection{Local Functions}

\section{MIS of the Email Notification Module} \label{EmailNotification}

\subsection{Module}

NotificationModule, User, PatientInfo, Trial

\subsection{Uses}


\subsection{Syntax}

\subsubsection{Exported Constants}
None

\subsubsection{Exported Access Programs}

\begin{center}
\begin{tabular}{p{4cm} p{4cm} p{4cm} p{4cm}}
\hline
\textbf{Name} & \textbf{In} & \textbf{Out} & \textbf{Exceptions} \\
\hline
sendEmail & String, String & - & DeliveryFailed \\
\hline
getAPIKey & - & String & - \\
\hline
setAPIKey & String & - & - \\
\hline
\end{tabular}
\end{center}

\subsection{Semantics}

\subsubsection{State Variables}
APIKey: String

\subsubsection{Environment Variables}
connection: API connection

\subsubsection{Assumptions}
\begin{itemize}
  \item Emailer API is operational and accessible
  \item ClinicalTrial.gov API is operational and accessible
\end{itemize}

\subsubsection{Access Routine Semantics}

\noindent sendEmail(emailAddress: String, subject: String, body: String):
\begin{itemize}
\item transition: None
\item output: out := email request sent through the emailing API
\item exception: exc := emailer returns error code $\rightarrow DeliveryFailed$
\end{itemize}

\noindent getAPIKey():
\begin{itemize}
\item transition: None
\item output: out := self.APIKey
\item exception: None
\end{itemize}

\noindent setAPIKey(key: String):
\begin{itemize}
\item transition: self.APIKey := key
\item output: None
\item exception: None
\end{itemize}


\subsubsection{Local Functions}
\noindent findNewTrials():
\begin{itemize}
\item transition: None
\item output: out := $\{trials : trial.postedDate > lastCheckedDate\}$
\item exception: None
\end{itemize}

\noindent matchUsersToNewTrials():
\begin{itemize}
\item transition: None
\item output: out := $\{user \times trial : user.conditions \subseteq trial.conditions\}$
\item exception: None
\end{itemize}


\section{MIS of Patient Data Collection Form Module} \label{PatientDataForm}

\subsection{Module}

PatientForm

\subsection{Uses}
InfoProfile, User

\subsection{Syntax}

\subsubsection{Exported Constants}
None

\subsubsection{Exported Access Programs}

\begin{center}
\begin{tabular}{p{4cm} p{4cm} p{4cm} p{4cm}}
\hline
\textbf{Name} & \textbf{In} & \textbf{Out} & \textbf{Exceptions} \\
\hline
displayPatientForm & - & ReactComponent & - \\
\hline
\end{tabular}
\end{center}

\subsection{Semantics}

\subsubsection{State Variables}

patientForm: ReactComponent \\
name: seq of char \\
dateOfBirth: seq of char \\
address: seq of char \\
gender: seq of char \\
healthDetails: TS Object$<$seq of char: Any$>$

\subsubsection{Environment Variables}

window \\
keyboard \\
mouse

\subsubsection{Assumptions}

None

\subsubsection{Access Routine Semantics}

\noindent displayPatientForm():
\begin{itemize}
\item transition: N/A
\item output: out := self.patientForm 
\item exception: N/A
\end{itemize}

\subsubsection{Local Functions}

\noindent submitPatientDetails(newPatientDetails: TS Object$<$seq of char: Any$>$):
\begin{itemize}
\item transition: self.healthDetails = newPatientDetails
\item output: N/A
\item exception: $exc := length(newPatientDetails) == 0 \Rightarrow InvalidDetails$
\end{itemize}


\section{MIS of User Login Form Module} \label{LoginForm}

\subsection{Module}

LoginForm

\subsection{Uses}
Login

\subsection{Syntax}

\subsubsection{Exported Constants}
None

\subsubsection{Exported Access Programs}

\begin{center}
\begin{tabular}{p{4cm} p{4cm} p{4cm} p{4cm}}
\hline
\textbf{Name} & \textbf{In} & \textbf{Out} & \textbf{Exceptions} \\
\hline
displayLoginForm & - & ReactComponent & - \\
\hline
\end{tabular}
\end{center}

\subsection{Semantics}

\subsubsection{State Variables}

loginForm: ReactComponent \\
emailAddress: seq of char \\
password: seq of char

\subsubsection{Environment Variables}

window \\
keyboard \\
mouse

\subsubsection{Assumptions}

None

\subsubsection{Access Routine Semantics}

\noindent displayLoginForm():
\begin{itemize}
\item transition: N/A
\item output: out := self.loginForm 
\item exception: N/A
\end{itemize}

\subsubsection{Local Functions}

\noindent submitLoginDetails(newEmailAddress: seq of char, newPassword: seq of char):
\begin{itemize}
\item transition: self.emailAddress, self.password = newEmailAddress, newPassword
\item output: N/A
\item exception: $exc := length(newEmailAddress) == 0 \vee length(newPassword) == 0 \Rightarrow InvalidLoginDetails$
\end{itemize}



\newpage

\bibliographystyle {plainnat}
\bibliography {../../../refs/References}

\newpage

\section{Appendix} \label{Appendix}

\wss{Extra information if required}

\end{document}