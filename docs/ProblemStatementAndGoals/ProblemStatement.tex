\documentclass{article}

\usepackage{tabularx}
\usepackage{booktabs}
\usepackage{float}

\title{Problem Statement and Goals\\\progname}

\author{\authname}

\date{}

%% Comments

\usepackage{color}

\newif\ifcomments\commentstrue %displays comments
%\newif\ifcomments\commentsfalse %so that comments do not display

\ifcomments
\newcommand{\authornote}[3]{\textcolor{#1}{[#3 ---#2]}}
\newcommand{\todo}[1]{\textcolor{red}{[TODO: #1]}}
\else
\newcommand{\authornote}[3]{}
\newcommand{\todo}[1]{}
\fi

\newcommand{\wss}[1]{\authornote{blue}{SS}{#1}} 
\newcommand{\plt}[1]{\authornote{magenta}{TPLT}{#1}} %For explanation of the template
\newcommand{\an}[1]{\authornote{cyan}{Author}{#1}}

%% Common Parts

\newcommand{\progname}{Software Engineering} % PUT YOUR PROGRAM NAME HERE
\newcommand{\authname}{Team 14, Reach
\\ Aamina Hussain
\\ David Moroniti
\\ Anika Peer
\\ Deep Raj
\\ Alan Scott} % AUTHOR NAMES                  

\usepackage{hyperref}
    \hypersetup{colorlinks=true, linkcolor=blue, citecolor=blue, filecolor=blue,
                urlcolor=blue, unicode=false}
    \urlstyle{same}
                                


\begin{document}

\maketitle

\begin{table}[hp]
\caption{Revision History} \label{TblRevisionHistory}
\begin{tabularx}{\textwidth}{llX}
\toprule
\textbf{Date} & \textbf{Developer(s)} & \textbf{Change}\\
\midrule
2023-09-23 & Alan Scott & Added goals and stretch goals\\
2023-09-24 & Alan Scott & Added additional stretch goal \\
2023-09-24 & Aamina Hussain & Added problem statement \\
2023-09-25 & Alan Scott & Updated goals based on meeting with supervisor \\
\bottomrule
\end{tabularx}
\end{table}

\section{Problem Statement}

% \wss{You should check your problem statement with the
% \href{https://github.com/smiths/capTemplate/blob/main/docs/Checklists/ProbState-Checklist.pdf}
% {problem statement checklist}.}
% \wss{You can change the section headings, as long as you include the required information.}

\subsection{Problem}
Clinical trials are a great opportunity for people to access alternate forms of treatment when they do not benefit 
from conventional or readily available treatment options. However, not many of these patients are able to take part 
in these clinical trials. This is mostly due to there being no direct path or connection between the researchers, patients, 
and healthcare providers. The lack of connection makes it difficult for patients to learn about possible trials they could 
qualify for, while also making it difficult for practitioners to find patients to take part in their studies.

\subsection{Proposed Solution}
Reach proposes to solve this problem by developing a web application that will be accessible by both practitioners 
and patients. The web application will use existing repositories of active research studies which will allow patients to 
have better access to clinical trials and make it easier for practitioners to find potential participants and match them to 
studies they are eligible for.

\subsection{Inputs and Outputs}

% \wss{Characterize the problem in terms of ``high level'' inputs and outputs.  
% Use abstraction so that you can avoid details.}
Inputs:
\begin{itemize}
    \item Requirements and constraints about active research studies
    \item Participants' necessary medical and demographic information
\end{itemize}

\noindent Outputs:
\begin{itemize}
    \item A match between the participant and the studies they are eligible for, if any
    \item An analysis of which demographic of patients are more interested in taking part in a clinical trial
\end{itemize}

\subsection{Stakeholders}
\begin{itemize}
    \item People who wish to take part in a clinical trial
	\item Practitioners/researchers who need to find potential participants to take part in their clinical trial/research studies
	\item Healthcare providers who would like their patients to have the opportunity to take part in a clinical trial if they are not benefitting from conventional treatment options
    \item Dr. Terence Ho and Dr. Ciaran Scallan (project supervisors)
\end{itemize}

\subsection{Environment}
Software: A web application with a user-friendly interface that can be accessed by both participants and practitioners.\\
\\
Hardware: A computer or smartphone that the user can use to access the web application.

% \wss{Hardware and software}

\section{Goals}

\begin{table}[H]
\centering
\begin{tabular}{| p{3cm} | p{5cm} | p{5cm} |}
\hline
Goal & Explanation & Reasoning \\
\hline  \hline
Data Collection & The system should collect personal and medical information from the client. This data will be stored in a remote repository. & The data will need to be accessible by the system to allow for the matching of patients to studies. \\
\hline
Reliant on Existing Repositories & The application should pull the details of studies from an existing repository of studies. & Since the application does not contain the studies themselves, the studies will need to be pulled from an external repository. \\
\hline
Remote Access & Users should be able to access the application through a web application over the internet, regardless of location. & Making the application readily available to the end users over the internet will maximize the accessibility of the application and improves the end user experience. \\
\hline
Security & The application should store user data in a manner such that it is only accessible by the intended parties. & While the data stored should be anonymized, the medical information is still sensitive and should therefore be protected. \\
\hline
Ease of Use & The user interface should be easy to understand by patients and clinicians alike. The interface should be intuitive to the point where users do not need to be taught how to use it. & The interface will be used by end users with a varied range of technical ability. Therefore, the interface should be as easy to use as possible. \\
\hline
Multiple Views & There should be different views for the application interface depending on the user group to which a user belongs. & Patients and clinicians will have different uses for the application, and will therefore need separate views for their differing use cases. \\
\hline
Speed & The application should load and react to user input quickly. &  Slow loading times will negatively impact the end user experience, potentially to the point of them leaving the site. Fast loading times contribute to positive user experience. \\
\hline

\end{tabular}
\end{table}

\section{Stretch Goals}

\begin{table}[H]
\centering
\begin{tabular}{| p{3cm} | p{5cm} | p{5cm} |}
\hline
Goal & Explanation & Reasoning \\
\hline  \hline
Expanded Search & The application should allow users to search for studies from multiple repositories. & The application should be scalable to include other repositories to pull studies from additional or external sources. \\
\hline
System Availability & The system should be virtually always available to the end users. & Maximizing uptime will make the system more accessible to users and increase the quality of the end user experience. \\
\hline
Notifications & The application should be able to send email reminders to patients. & Email reminders will ensure that patients do not miss their appointments, and that studies are not missing patients. \\
\hline

\end{tabular}
\end{table}

\end{document}