\documentclass[12pt, titlepage]{article}

\usepackage{booktabs}
\usepackage{tabularx}
\usepackage{hyperref}
\hypersetup{
    colorlinks,
    citecolor=black,
    filecolor=black,
    linkcolor=red,
    urlcolor=blue
}
\usepackage[round]{natbib}
\newcounter{NFR_Counter}
\addtocounter{NFR_Counter}{1}
\newcounter{FR_Counter}
\addtocounter{FR_Counter}{1}

\title{Software Requirements Specification\\\progname}

\author{\authname}

\date{}

%% Comments

\usepackage{color}

\newif\ifcomments\commentstrue %displays comments
%\newif\ifcomments\commentsfalse %so that comments do not display

\ifcomments
\newcommand{\authornote}[3]{\textcolor{#1}{[#3 ---#2]}}
\newcommand{\todo}[1]{\textcolor{red}{[TODO: #1]}}
\else
\newcommand{\authornote}[3]{}
\newcommand{\todo}[1]{}
\fi

\newcommand{\wss}[1]{\authornote{blue}{SS}{#1}} 
\newcommand{\plt}[1]{\authornote{magenta}{TPLT}{#1}} %For explanation of the template
\newcommand{\an}[1]{\authornote{cyan}{Author}{#1}}

%% Common Parts

\newcommand{\progname}{Software Engineering} % PUT YOUR PROGRAM NAME HERE
\newcommand{\authname}{Team 14, Reach
\\ Aamina Hussain
\\ David Moroniti
\\ Anika Peer
\\ Deep Raj
\\ Alan Scott} % AUTHOR NAMES                  

\usepackage{hyperref}
    \hypersetup{colorlinks=true, linkcolor=blue, citecolor=blue, filecolor=blue,
                urlcolor=blue, unicode=false}
    \urlstyle{same}
                                


\begin{document}

\maketitle

\pagenumbering{roman}
\tableofcontents
\listoftables
\listoffigures

\begin{table}[bp]
\caption{\bf Revision History}
\begin{tabularx}{\textwidth}{p{3cm}p{2cm}X}
\toprule {\bf Date} & {\bf Version} & {\bf Notes}\\
\midrule
Date 1 & 1.0 & Notes\\
Date 2 & 1.1 & Notes\\
\bottomrule
\end{tabularx}
\end{table}

\newpage

\pagenumbering{arabic}

This document describes the requirements for ....  The template for the Software
Requirements Specification (SRS) is a subset of the Volere
template~\citep{RobertsonAndRobertson2012}.  If you make further modifications
to the template, you should explicitly state what modifications were made.

\section{Project Drivers}

\subsection{The Purpose of the Project}

\subsection{The Stakeholders}

\subsubsection{The Client}

\subsubsection{The Customers}

\subsubsection{Other Stakeholders}

\subsection{Mandated Constraints}

\subsection{Naming Conventions and Terminology}

\begin{itemize}
    \item Platform - The web system/application.
    \item Heavy traffic - When many users are accessing the platform at one time.
\end{itemize}

\subsection{Relevant Facts and Assumptions}

User characteristics should go under assumptions.

\section{Functional Requirements}

\subsection{The Scope of the Work and the Product}

\subsubsection{The Context of the Work}

\subsubsection{Work Partitioning}

\subsubsection{Individual Product Use Cases}

\subsection{Functional Requirements}

\section{Non-functional Requirements}

\subsection{Look and Feel Requirements}

\subsection{Usability and Humanity Requirements}

\textbf{NFR-\the\value{NFR_Counter}:}
The application interface shall be intuitive to use to a point where users do not require a tutorial or help from another person. \\
\textbf{Rationale:}
The interface will be used by users with a wide range of technical ability, so making the interface as easy to learn as possible will make the application maximally accessible. \\
\textbf{Fit criterion:}
95\% of users tested should be able to accomplish the main function of the application without need for outside intervention. \\
\textbf{Traceability:}
Traces to functional requirements involving the user interface. \\~\\
\addtocounter{NFR_Counter}{1}

\noindent\textbf{NFR-\the\value{NFR_Counter}:}
The application interface shall only include the minimum necessary elements for the system to function effectively. \\
\textbf{Rationale:}
The simplicity of the system is important to ensure that the system is easy for all users to interact with. \\
\textbf{Fit criterion:}
Testers should not be able to identify any element of the user interface that does not serve any immediate and apparent use. \\
\textbf{Traceability:}
Traces to functional requirements involving the user interface. \\~\\
\addtocounter{NFR_Counter}{1}

\noindent{\textbf{NFR-\the\value{NFR_Counter}:}}
The text and image elements of the user interface should be large enough such that  \\
\textbf{Rationale:}
Due to the nature of the system, it can be expected that users of varying visual aptitude will be interacting with the user interface. Making the interface visually accessible will ensure that users of non-perfect optical perscriptions are not excluded from its use. \\
\textbf{Fit criterion:}
A user with a prescription of less than 4 diopters should be able to effectively use the interface without the use of perscription glasses. Alternatively, a perfectly sighted individual should be able to read the interface 1 meter away from their computer monitor without issue. \\
\textbf{Traceability:}
 \\~\\
\addtocounter{NFR_Counter}{1}


\subsection{Performance Requirements}

\textbf{NFR-\the\value{NFR_Counter}:}
The system shall load and display trials to the user in a timely manner.\\
\textbf{Rationale:}
In addition to obvious usability reasons, there is a potential for users (especially clinicians) to rapidly
change the criteria for eligibility, meaning the time spent waiting for trials to load can grow very fast if it is slow.\\
\textbf{Fit criterion:}
The system shall not take longer than 2 seconds to load and display a set of trials, during any given search made by a user.\\
\textbf{Traceability:}
**Traces to functional requirements related to searching for trials, displaying trials, getting data from repositories.** \\~\\
\addtocounter{NFR_Counter}{1}

\noindent{\textbf{NFR-\the\value{NFR_Counter}:}}
The system shall remain performant during times of heavy traffic.\\
\textbf{Rationale:}
The system will experience varying amounts of traffic, and it should be prepared to handle all cases (heavy traffic being the 
one to likely cause issues).\\
\textbf{Fit criterion:}
Any system API call shall take less than 1 second to respond for up to 1000 concurrent users, and less than 2 seconds to respond for anything 
greater (assuming the system will not see $>10000$ users).\\
\textbf{Traceability:}
Traces to pretty much all FRs that are mentioned in the other performance requirements \\~\\
\addtocounter{NFR_Counter}{1}

\noindent{\textbf{NFR-\the\value{NFR_Counter}:}}
The systems UI elements shall acknowledge all forms of user input in a timely manner.\\
\textbf{Rationale:}
Timely response from the system is necessary for a good user experience. Additionally, things like keyboard
shortcuts should be just as performant as using a mouse (and vice-versa).\\
\textbf{Fit criterion:}
The system shall take less than 150ms to acknowledge user input, and make it clear to the user that it has registered the request.\\
\textbf{Traceability:}
Traces to any FR that requires user input of some form \\~\\
\addtocounter{NFR_Counter}{1}

\noindent{\textbf{NFR-\the\value{NFR_Counter}:}}
The system shall efficiently store/retrieve user data from the database.\\
\textbf{Rationale:}
For the system to remain performant, it must handle data efficiently, especially as the total number of users increase.\\
\textbf{Fit criterion:}
The system shall take no longer than 500ms when querying the database for user info (includes query execution time + any overhead of executing 
the query).\\
\textbf{Traceability:}
Traces to any FR that requires user information \\~\\
\addtocounter{NFR_Counter}{1}

\noindent{\textbf{NFR-\the\value{NFR_Counter}:}}
The system shall be available to users 99\% of the time.\\
\textbf{Rationale:}
Since users will likely make infrequent visits to the platform, it is necessary that the platform will be available at those 
times (i.e., if the system is frequently unavailable, it increases the chances that users will never use the platform).\\
\textbf{Fit criterion:}
The system shall experience no more than 15 hours of downtime throughout the year.\\
\textbf{Traceability:}
N/A
\addtocounter{NFR_Counter}{1}

\subsection{Operational and Environmental Requirements}

\subsection{Maintainability and Support Requirements}


\noindent{\textbf{NFR-\the\value{NFR_Counter}:}}
The system shall be easy to fix in the event it experiences unexpected downtime.\\
\textbf{Rationale:}
While it may not happen often, an unexpected system failure is definitely possible. As the time it takes 
to fix the issue increases, so does the number of users negatively affected.\\
\textbf{Fit criterion:}
On average, it should take less than 1 hour to restore the system in the event of a failure.\\
\textbf{Traceability:}
N/A \\~\\
\addtocounter{NFR_Counter}{1}


\noindent{\textbf{NFR-\the\value{NFR_Counter}:}}
The system shall be adaptable to new requirements.\\
\textbf{Rationale:}
It is likely new requirements will be discovered once users begin to use the platform, and there will be a need/desire to 
implement these quickly and effectively.\\
\textbf{Fit criterion:}
The system must satisfy all of its existing requirements after the implementation of a new requirement/feature.\\
\textbf{Traceability:}
N/A
\addtocounter{NFR_Counter}{1}


\subsection{Security Requirements}
\noindent\textbf{NFR-\the\value{NFR_Counter}:}
test
\addtocounter{NFR_Counter}{1}

\subsection{Cultural Requirements}
\noindent\textbf{NFR-\the\value{NFR_Counter}:}
The interface shall be useable by users who speak a language other than English or French.  \\
\textbf{Rationale:}
Restricting the interface's language to only official languages will exclude many ethnic minorities from being able to access clinical studies. Making the program multilingual will ensure that a minimal number of ethnic minorities will be excluded. \\
\textbf{Fit criterion:}
Users who speak the languages in the top 5 most spoken non-official languages in Canada should be able to use the interface without issue. \\
\textbf{Traceability:}
Traces to functional requirements involving the user interface. \\~\\
\addtocounter{NFR_Counter}{1}

\subsection{Legal Requirements}
\noindent\textbf{NFR-\the\value{NFR_Counter}:}
The system shall be compliant with the \textit{Personal Health Information Privacy Act, 2004}  \\
\textbf{Rationale:}
Since the data stored pertains to personal medical information, we are legally required to manage the data in such a way that protects the personal information of the patients. \\
\textbf{Fit criterion:}
The system should be able to pass an independent audit to ensure the data is being collected and secured in an appropriate manner. \\
\textbf{Traceability:}
Traces to requirements involving user input (specifically medical data input) and security requirements. \\~\\
\addtocounter{NFR_Counter}{1}

\subsection{Health and Safety Requirements}

This section is not in the original Volere template, but health and safety are
issues that should be considered for every engineering project.

\section{Project Issues}

\subsection{Open Issues}

\subsection{Off-the-Shelf Solutions}

\subsection{New Problems}

\subsection{Tasks}

\subsection{Migration to the New Product}

\subsection{Risks}

\subsection{Costs}

\subsection{User Documentation and Training}

\subsection{Waiting Room}

N/A

\subsection{Ideas for Solutions}

\bibliographystyle{plainnat}

\bibliography{SRS}

\newpage

\section{Appendix}

This section has been added to the Volere template.  This is where you can place
additional information.

\subsection{Symbolic Parameters}

The definition of the requirements will likely call for SYMBOLIC\_CONSTANTS.
Their values are defined in this section for easy maintenance.


\end{document}