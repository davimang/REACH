\documentclass[12pt, titlepage]{article}

\usepackage{booktabs}
\usepackage{tabularx}
\usepackage{hyperref}
\hypersetup{
    colorlinks,
    citecolor=black,
    filecolor=black,
    linkcolor=red,
    urlcolor=blue
}
\usepackage[round]{natbib}
\newcounter{NFR_Counter}
\addtocounter{NFR_Counter}{1}
\newcounter{FR_Counter}
\addtocounter{FR_Counter}{1}

\title{Software Requirements Specification\\\progname}

\author{\authname}

\date{}

%% Comments

\usepackage{color}

\newif\ifcomments\commentstrue %displays comments
%\newif\ifcomments\commentsfalse %so that comments do not display

\ifcomments
\newcommand{\authornote}[3]{\textcolor{#1}{[#3 ---#2]}}
\newcommand{\todo}[1]{\textcolor{red}{[TODO: #1]}}
\else
\newcommand{\authornote}[3]{}
\newcommand{\todo}[1]{}
\fi

\newcommand{\wss}[1]{\authornote{blue}{SS}{#1}} 
\newcommand{\plt}[1]{\authornote{magenta}{TPLT}{#1}} %For explanation of the template
\newcommand{\an}[1]{\authornote{cyan}{Author}{#1}}

%% Common Parts

\newcommand{\progname}{Software Engineering} % PUT YOUR PROGRAM NAME HERE
\newcommand{\authname}{Team 14, Reach
\\ Aamina Hussain
\\ David Moroniti
\\ Anika Peer
\\ Deep Raj
\\ Alan Scott} % AUTHOR NAMES                  

\usepackage{hyperref}
    \hypersetup{colorlinks=true, linkcolor=blue, citecolor=blue, filecolor=blue,
                urlcolor=blue, unicode=false}
    \urlstyle{same}
                                


\begin{document}

\maketitle

\pagenumbering{roman}
\tableofcontents
\listoftables
\listoffigures

\begin{table}[bp]
\caption{\bf Revision History}
\begin{tabularx}{\textwidth}{p{3cm}p{2cm}X}
\toprule {\bf Date} & {\bf Version} & {\bf Notes}\\
\midrule
Date 1 & 1.0 & Notes\\
Date 2 & 1.1 & Notes\\
\bottomrule
\end{tabularx}
\end{table}

\newpage

\pagenumbering{arabic}

This document describes the requirements for ....  The template for the Software
Requirements Specification (SRS) is a subset of the Volere
template~\citep{RobertsonAndRobertson2012}.  If you make further modifications
to the template, you should explicitly state what modifications were made.

\section{Project Drivers}

\subsection{The Purpose of the Project}

\subsection{The Stakeholders}

\subsubsection{The Client}

\subsubsection{The Customers}

\subsubsection{Other Stakeholders}

\subsection{Mandated Constraints}

\subsection{Naming Conventions and Terminology}

\begin{itemize}
    \item Platform - The web system/application.
    \item Heavy traffic - When many users are accessing the platform at one time.
\end{itemize}

\subsection{Relevant Facts and Assumptions}

User characteristics should go under assumptions.

\section{Functional Requirements}

\subsection{The Scope of the Work and the Product}

\subsubsection{The Context of the Work}

\subsubsection{Work Partitioning}

\subsubsection{Individual Product Use Cases}

\subsection{Functional Requirements}

\section{Non-functional Requirements}

\subsection{Look and Feel Requirements}

\subsection{Usability and Humanity Requirements}

\subsection{Performance Requirements}

\textbf{NFR-\the\value{NFR_Counter}:}
The system shall load and display trials to the user in a timely manner.\\
\textbf{Rationale:}
In addition to obvious usability reasons, there is a potential for users (especially clinicians) to rapidly
change the criteria for eligibility, meaning the time spent waiting for trials to load can grow very fast if it is slow.\\
\textbf{Fit criterion:}
The system shall not take longer than 2 seconds to load and display a set of trials, during any given search made by a user.\\
\textbf{Traceability:}
**Traces to functional requirements related to searching for trials, displaying trials, getting data from repositories.** \\~\\
\addtocounter{NFR_Counter}{1}

\noindent{\textbf{NFR-\the\value{NFR_Counter}:}}
The system shall remain performant during times of heavy traffic.\\
\textbf{Rationale:}
The system will experience varying amounts of traffic, and it should be prepared to handle all cases (heavy traffic being the 
one to likely cause issues).\\
\textbf{Fit criterion:}
The system shall satisfy all other performance requirements/criteria mentioned, for up to 1000 concurrent users.\\
\textbf{Traceability:}
Traces to pretty much all FRs that are mentioned in the other performance requirements \\~\\
\addtocounter{NFR_Counter}{1}

\noindent{\textbf{NFR-\the\value{NFR_Counter}:}}
The systems UI elements shall acknowledge all forms of user input in a timely manner.\\
\textbf{Rationale:}
Timely response from the system is necessary for a good user experience. Additionally, things like keyboard
shortcuts should be just as performant as using a mouse (and vice-versa).\\
\textbf{Fit criterion:}
The system shall take less than 150ms to acknowledge user input, and make it clear to the user that it has registered the request.\\
\textbf{Traceability:}
Traces to any FR that requires user input of some form \\~\\
\addtocounter{NFR_Counter}{1}


\noindent{\textbf{NFR-\the\value{NFR_Counter}:}}
The system shall efficiently store/retrieve user data from the database.\\
\textbf{Rationale:}
For the system to remain performant, it must handle data efficiently, especially as the total number of users increase.\\
\textbf{Fit criterion:}
The system shall take no longer than 500ms when querying the database for user info (includes query execution time + any overhead of executing 
the query).\\
\textbf{Traceability:}
Traces to any FR that requires user information \\~\\
\addtocounter{NFR_Counter}{1}

\subsection{Operational and Environmental Requirements}

\subsection{Maintainability and Support Requirements}

\subsection{Security Requirements}

\subsection{Cultural Requirements}

\subsection{Legal Requirements}

\subsection{Health and Safety Requirements}

This section is not in the original Volere template, but health and safety are
issues that should be considered for every engineering project.

\section{Project Issues}

\subsection{Open Issues}

\subsection{Off-the-Shelf Solutions}

\subsection{New Problems}

\subsection{Tasks}

\subsection{Migration to the New Product}

\subsection{Risks}

\subsection{Costs}

\subsection{User Documentation and Training}

\subsection{Waiting Room}

\subsection{Ideas for Solutions}

\bibliographystyle{plainnat}

\bibliography{SRS}

\newpage

\section{Appendix}

This section has been added to the Volere template.  This is where you can place
additional information.

\subsection{Symbolic Parameters}

The definition of the requirements will likely call for SYMBOLIC\_CONSTANTS.
Their values are defined in this section for easy maintenance.


\end{document}