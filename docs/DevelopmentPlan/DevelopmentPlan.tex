\documentclass{article}

\usepackage{booktabs}
\usepackage{tabularx}

\title{Development Plan\\\progname}

\author{\authname}

\date{}

\input{../Comments}
%% Common Parts

\newcommand{\progname}{Software Engineering} % PUT YOUR PROGRAM NAME HERE
\newcommand{\authname}{Team 14, Reach
\\ Aamina Hussain
\\ David Moroniti
\\ Anika Peer
\\ Deep Raj
\\ Alan Scott} % AUTHOR NAMES                  

\usepackage{hyperref}
    \hypersetup{colorlinks=true, linkcolor=blue, citecolor=blue, filecolor=blue,
                urlcolor=blue, unicode=false}
    \urlstyle{same}
                                


\begin{document}

\maketitle

\begin{table}[hp]
\caption{Revision History} \label{TblRevisionHistory}
\begin{tabularx}{\textwidth}{llX}
\toprule
\textbf{Date} & \textbf{Developer(s)} & \textbf{Change}\\
\midrule
09/22/23 & David & Added first draft for tech and coding standard\\
09/23/23 & Deep & Added first draft for proof of concept plan and project schedule\\
09/23/23 & Anika & Added intro and sections 1 - 4 of Development Plan\\
... & ... & ...\\
\bottomrule
\end{tabularx}
\end{table}

\wss{Put your introductory blurb here, Fill in names.}
The following document outlines the development plan for the BLAH system.
As the scope of the project is further elborated upon, 
this document will be revised in order to reflect the most updated plan for the project.\\


\section{Team Meeting Plan} 
Team meetings will occur at least once a week, during the scheduled tutorial time Mondays from 2:30 PM - 4:30 PM ET.
All members are expected to attend these meetings, that will be referred to as development (dev) team syncs. 
If a member has a timing conflict, they must notify the rest of the team as soon as possible, in order to allow for recheduling.
Any additional dev team syncs will occur based on member schedule and are not mandatory unless otherwise specified.\\

On a biweekly basis, the team will meet with the project supervisors to discuss progress and gain clarity on any roadblocks, this is called the stakeholder sync. 
The timing for these meetings depends on the avialability of the project supervisors and can be subject to change. 
Although attendance for the stakeholder sync is not mandatory, it is highly encouraged. 
As with the dev sync, team members should notify the team of any scheduling conflicts although rescheduling may not be possible due to the busy schedules of project supervisors. \\

Depending on team availability, meetings may occur on campus or virtually. 
Given the travel time for certain team members, virtual meetings are preffered.\\

Where agendas are concerned, all team members are expected to contribute to the agenda prior to the meeting. 
Any concerns, questions, updates or roadblocks should be jotted in point form on the agenda and can be further discussed in meeting. 
All agendas will be tracked as issues in the project repository and will be closed after meetings. 




\section{Team Communication Plan}

\section{Team Member Roles}

\section{Workflow Plan}

\begin{itemize}
	\item How will you be using git, including branches, pull request, etc.?
	\item How will you be managing issues, including template issues, issue
	classificaiton, etc.?
\end{itemize}

\section{Proof of Concept Demonstration Plan}

The main risks for the success of this project are; none of the members of the team have deployed a web app from scratch before, and that the data provided by the clinical is often in natural language. Since none of the team members have deployed a web app from scratch this may be a risk as we are not certain of all the steps which must be taken to publish a web app. The data provided by the clinical trials being in natural language also is a risk as we will have to research and use a natural language processor to extract the relevant data points. 

Therfore to demonstate that we can overcome these risks our proof of concept demonstations will have to both be deployed and able to be ascessed from a device that is not running the source code, and the application must be able to extract the requirements for the clinical trial from the data source. Also the application should be able to display the retreived requiremnts in a GUI of some fashion.

\section{Technology}

As a result of our system being a pretty complex full-stack application, it will be built
using several different tools and technologies, across both the frontend and the backend.
\\~\\
\textbf{Frontend}

\begin{itemize}
	\item Programming language - Typescript
	\item Unit testing framework - Jest
	\item Frontend web framework - React\\
\end{itemize}

\textbf{Backend}

\begin{itemize}
	\item Programming language - Python
	\item Linter - Flake8, Mypy, autoflake
	\item Unit testing framework - pytest
	\item Backend web framework(s) - Django, FastAPI\\
\end{itemize}

In addition to the technologies mentioned above, we will be using a few tools that will help us 
develop our app more efficiently and effectively. First, we will containerize our app
by using docker, which will make it easier to test and deploy our application (also use of docker desktop for running/testing
our application locally). Furthermore, for our continous 
integration pipeline, we will be using github actions to run automated tests and linting for both the frontend and backend 
services/code, and to build our docker images. Finally, we plan to use firebase as our cloud environment, which
is where our application will actually run. \\

There are also a few tools/technologies that still need to be decided upon. We will need to have some database to store
client information. It is likely that this will be a relational database, and will be one of Postgres, MySQL, or SQLite.
Additionally, it is possible that we will need to use some existing library/tool for parsing trial eligibility criteria, however this 
will become more clear in the following weeks (i.e., an advanced parsing library, or existing NLP designed for healthcare systems).

\section{Coding Standard}

For all of our python code/services, we will be following the Pep8 coding standard, and we will be enforcing this 
with the help of the flake8 linter. We will also be using mypy, which will enforce strict typing (which is usually not present/mandatory
in python), meaning all of our function parameters, return types, class variables, etc.. will need to have its type defined. Finally, we will
have docstrings present in each module, to give a brief description of what the main purpose of the module is.\\

For the frontend code we will not be following any formal coding standard, but we will be following best practices
when coding in typescript and when using React. Some of these best practices include - 

\begin{itemize}
	\item use functional components
	\item don't use inline-styles
	\item maintain a proper import structure (third-party imports first --> internal imports below)
\end{itemize}

\section{Project Scheduling}

\begin{itemize}
	\item September 25 - Problem Statement, Proof of Concept Plan, Development Plan
	\item October 4 - Requirements Document
	\item October 20 - Hazard Analysis
	\item November 3 - Verification and Validation Plan
	\item November 13-24 - Proof of Concept Demo
	\item January 17 - Design Documentation
	\item February 5-16 - Revision 0 Presentation
	\item March 6 - Verification and Validation Report
	\item March 18-29 - Final Demonstation
	\item April 4 - Final Documentation
	\item April 9 - Capstone Expo
\end {itemize}

\end{document}