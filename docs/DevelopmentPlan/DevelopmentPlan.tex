\documentclass{article}

\usepackage{booktabs}
\usepackage{tabularx}

\title{Development Plan\\\progname}

\author{\authname}

\date{}

%% Comments

\usepackage{color}

\newif\ifcomments\commentstrue %displays comments
%\newif\ifcomments\commentsfalse %so that comments do not display

\ifcomments
\newcommand{\authornote}[3]{\textcolor{#1}{[#3 ---#2]}}
\newcommand{\todo}[1]{\textcolor{red}{[TODO: #1]}}
\else
\newcommand{\authornote}[3]{}
\newcommand{\todo}[1]{}
\fi

\newcommand{\wss}[1]{\authornote{blue}{SS}{#1}} 
\newcommand{\plt}[1]{\authornote{magenta}{TPLT}{#1}} %For explanation of the template
\newcommand{\an}[1]{\authornote{cyan}{Author}{#1}}

%% Common Parts

\newcommand{\progname}{Software Engineering} % PUT YOUR PROGRAM NAME HERE
\newcommand{\authname}{Team 14, Reach
\\ Aamina Hussain
\\ David Moroniti
\\ Anika Peer
\\ Deep Raj
\\ Alan Scott} % AUTHOR NAMES                  

\usepackage{hyperref}
    \hypersetup{colorlinks=true, linkcolor=blue, citecolor=blue, filecolor=blue,
                urlcolor=blue, unicode=false}
    \urlstyle{same}
                                


\begin{document}

\maketitle

\begin{table}[hp]
\caption{Revision History} \label{TblRevisionHistory}
\begin{tabularx}{\textwidth}{llX}
\toprule
\textbf{Date} & \textbf{Developer(s)} & \textbf{Change}\\
\midrule
11/22/23 & David & Added first draft for tech and coding standard\\
Date2 & Name(s) & Description of changes\\
... & ... & ...\\
\bottomrule
\end{tabularx}
\end{table}

\wss{Put your introductory blurb here.}

\section{Team Meeting Plan}

\section{Team Communication Plan}

\section{Team Member Roles}

\section{Workflow Plan}

\begin{itemize}
	\item How will you be using git, including branches, pull request, etc.?
	\item How will you be managing issues, including template issues, issue
	classificaiton, etc.?
\end{itemize}

\section{Proof of Concept Demonstration Plan}

What is the main risk, or risks, for the success of your project?  What will you
demonstrate during your proof of concept demonstration to convince yourself that
you will be able to overcome this risk?

\section{Technology}

As a result of our system being a pretty complex full-stack application, it will be built
using several different tools and technologies, across both the frontend and the backend.
\\~\\
\textbf{Frontend}

\begin{itemize}
	\item Programming language - Typescript
	\item Unit testing framework - Jest
	\item Frontend web framework - React\\
\end{itemize}

\textbf{Backend}

\begin{itemize}
	\item Programming language - Python
	\item Linter - Flake8, Mypy, autoflake
	\item Unit testing framework - pytest
	\item Backend web framework(s) - Django, FastAPI\\
\end{itemize}

In addition to the technologies mentioned above, we will be using a few tools that will help us 
develop our app more efficiently and effectively. First, we will containerize our app
by using docker, which will make it easier to test and deploy our application (also use of docker desktop for running/testing
our application locally). Furthermore, for our continous 
integration pipeline, we will be using github actions to run automated tests and linting for both the frontend and backend 
services/code, and to build our docker images. Finally, we plan to use firebase as our cloud environment, which
is where our application will actually run. \\

There are also a few tools/technologies that still need to be decided upon. We will need to have some database to store
client information. It is likely that this will be a relational database, and will be one of Postgres, MySQL, or SQLite.
Additionally, it is possible that we will need to use some existing library/tool for parsing trial eligibility criteria, however this 
will become more clear in the following weeks (i.e., an advanced parsing library, or existing NLP designed for healthcare systems).

\section{Coding Standard}

For all of our python code/services, we will be following the Pep8 coding standard, and we will be enforcing this 
with the help of the flake8 linter. We will also be using mypy, which will enforce strict typing (which is usually not present/mandatory
in python), meaning all of our function parameters, return types, class variables, etc.. will need to have its type defined. Finally, we will
have docstrings present in each module, to give a brief description of what the main purpose of the module is.\\

For the frontend code we will not be following any formal coding standard, but we will be following best practices
when coding in typescript and when using React. Some of these best practices include - 

\begin{itemize}
	\item use functional components
	\item don't use inline-styles
	\item maintain a proper import structure (third-party imports first --> internal imports below)
\end{itemize}

\section{Project Scheduling}

\wss{How will the project be scheduled?}

\end{document}